\subsection{Case study: simple hash function}

(This piece of text was initially added to my ``Reverse Engineering for Beginners'' book (\url{beginners.re}) at March 2014)
\footnote{This example was also used by Murphy Berzish in his lecture about \ac{SAT} and \ac{SMT}:
\url{http://mirror.csclub.uwaterloo.ca/csclub/mtrberzi-sat-smt-slides.pdf},
\url{http://mirror.csclub.uwaterloo.ca/csclub/mtrberzi-sat-smt.mp4}}.

Amateur cryptography is usually (unintentionally) 
very weak and can be broken easily---for cryptographers, of course.

But let's pretend we are not among these crypto-professionals.

Here is one-way hash function, that converted a 64-bit value to another and we need to try to reverse its flow back.

\subsubsection{Manual decompiling}

Here its assembly language listing in IDA:

\lstinputlisting{crypto/hash/algo_1.asm}

The example was compiled by GCC, so the first argument is passed in ECX.

If you don't have Hex-Rays, or if you distrust to it, you can try to reverse this code manually.
One method is to represent the CPU registers as local C variables and replace each instruction by a one-line equivalent expression, like:

\lstinputlisting{crypto/hash/algo_2.c}

If you are careful enough, this code can be compiled and will even work in the same way as the original.

Then, we are going to rewrite it gradually, keeping in mind all registers usage.
Attention and focus is very important here---any tiny typo may ruin all your work!

Here is the first step:

\lstinputlisting{crypto/hash/algo_3.c}

Next step:

\lstinputlisting{crypto/hash/algo_4.c}

We can spot the division using multiplication.
Indeed, let's calculate the divider in Wolfram Mathematica:

\begin{lstlisting}[caption=Wolfram Mathematica]
In[1]:=N[2^(64 + 5)/16^^8888888888888889]
Out[1]:=60.
\end{lstlisting}

We get this:

\lstinputlisting{crypto/hash/algo_5.c}

One more step:

\lstinputlisting{crypto/hash/algo_6.c}

By simple reducing, we finally see that it's calculating the remainder, not the quotient:

\lstinputlisting{crypto/hash/algo_7.c}

We end up with this fancy formatted source-code:

\lstinputlisting{crypto/hash/algo_src.c}

Since we are not cryptoanalysts we can't find an easy way to generate the input value for some specific output value.
The rotate instruction's coefficients look frightening---it's a warranty that the function is not bijective,
it has collisions, or, speaking more simply, many inputs may be possible for one output.

Brute-force is not solution because values are 64-bit ones, that's beyond reality.

\subsubsection{Now let's use the Z3 \ac{SMT} solver}

Still, without any special cryptographic knowledge, we may try to break this 
algorithm using the excellent \ac{SMT} solver from Microsoft Research named Z3\footnote{\url{http://go.yurichev.com/17314}}.
It is in fact theorem prover, but we are going to use it as \ac{SMT} solver.
Simply said, we can think about it as a system capable of solving huge equation systems.

Here is the Python source code:

\lstinputlisting[numbers=left]{crypto/hash/1.py}

This is going to be our first solver.

We see the variable definitions on line 7.
These are just 64-bit variables.
\TT{i1..i6} are intermediate variables, representing the values in the registers between instruction executions.

Then we add the so-called constraints on lines 10..15.
The last constraint at 17 is the most important one: 
we are going to try to find an input value for which our algorithm will produce 10816636949158156260.

Essentially, the \ac{SMT}-solver searches for (any) values that satisfies all constraints.

RotateRight, RotateLeft, URem---are functions from the Z3 Python API, not related to Python language.

Then we run it:

\begin{lstlisting}
...>python.exe 1.py
sat
[i1 = 3959740824832824396,
 i3 = 8957124831728646493,
 i5 = 10816636949158156260,
 inp = 1364123924608584563,
 outp = 10816636949158156260,
 i4 = 14065440378185297801,
 i2 = 4954926323707358301]
 inp=0x12EE577B63E80B73
outp=0x961C69FF0AEFD7E4
\end{lstlisting}

``sat'' mean ``satisfiable'', i.e., the solver was able to find at least one solution.
The solution is printed in the square brackets.
The last two lines are the input/output pair in hexadecimal form.
Yes, indeed, if we run our function with \TT{0x12EE577B63E80B73} as input, the algorithm will produce the value we were looking for.

But, as we noticed before, the function we work with is not bijective, so there may be other correct input values.
The Z3 \ac{SMT} solver is not capable of producing more than one result, but let's hack our example slightly, 
by adding line 19, which implies ``look for any other results than this'':

\lstinputlisting[numbers=left]{crypto/hash/2.py}

Indeed, it finds another correct result:

\begin{lstlisting}
...>python.exe 2.py
sat
[i1 = 3959740824832824396,
 i3 = 8957124831728646493,
 i5 = 10816636949158156260,
 inp = 10587495961463360371,
 outp = 10816636949158156260,
 i4 = 14065440378185297801,
 i2 = 4954926323707358301]
 inp=0x92EE577B63E80B73
outp=0x961C69FF0AEFD7E4
\end{lstlisting}

This can be automated.
Each found result can be added as a constraint and then the next result will be searched for.
Here is a slightly more sophisticated example:

\lstinputlisting[numbers=left]{crypto/hash/3.py}

We got:

\begin{lstlisting}
1364123924608584563
1234567890
9223372038089343698
4611686019661955794
13835058056516731602
3096040143925676201
12319412180780452009
7707726162353064105
16931098199207839913
1906652839273745429
11130024876128521237
15741710894555909141
6518338857701133333
5975809943035972467
15199181979890748275
10587495961463360371
results total= 16
\end{lstlisting}

So there are 16 correct input values for \TT{0x92EE577B63E80B73} as a result.

The second is 1234567890---it is indeed the value which was used by me originally while preparing this example.

Let's also try to research our algorithm a bit more.
Acting on a sadistic whim, let's find if there are any possible input/output pairs in 
which the lower 32-bit parts are equal to each other?

Let's remove the \textit{outp} constraint and add another, at line 17:

\lstinputlisting[numbers=left]{crypto/hash/4.py}

It is indeed so:

\begin{lstlisting}
sat
[i1 = 14869545517796235860,
 i3 = 8388171335828825253,
 i5 = 6918262285561543945,
 inp = 1370377541658871093,
 outp = 14543180351754208565,
 i4 = 10167065714588685486,
 i2 = 5541032613289652645]
 inp=0x13048F1D12C00535
outp=0xC9D3C17A12C00535
\end{lstlisting}

Let's be more sadistic and add another constraint: last 16 bits must be \TT{0x1234}:

\lstinputlisting[numbers=left]{crypto/hash/5.py}

Oh yes, this possible as well:

\begin{lstlisting}
sat
[i1 = 2834222860503985872,
 i3 = 2294680776671411152,
 i5 = 17492621421353821227,
 inp = 461881484695179828,
 outp = 419247225543463476,
 i4 = 2294680776671411152,
 i2 = 2834222860503985872]
 inp=0x668EEC35F961234
outp=0x5D177215F961234
\end{lstlisting}

Z3 works very fast and it implies that the algorithm is weak, it is not cryptographic at all
(like the most of the amateur cryptography).

Is it possible to tackle real cryptography by these methods? 
Real algorithms like AES, RSA, etc, can also be represented as huge system of equations, 
but these are so huge that they are impossible to work with on computers, now or in the near future.
Of course, cryptographers are fully aware of this.

Summarizing, when dealing with amateur crypto, 
it's a very good idea to try a \ac{SMT}/\ac{SAT} solver (like Z3).

