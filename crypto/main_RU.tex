\section{(Любительская) криптография}
\label{crypto}

\subsection{\textit{Серьезная} криптография}

Вернемся к раннее использованнму методу (\ref{symbolic_exec}), чтобы сконструировать выражения используя запущенную
Питоновскую ф-цию.

Можно найти выражения для всех четырех выходов алгоритма шифрования XXTEA:

\lstinputlisting{crypto/xxtea.py}

Ключ выбирается в зависимости от входных данных, и, очевидно, мы не знаем его во время символьного исполнения,
так что мы оставляем выражение вроде \TT{k[...]}.

Теперь результаты для одного раунда, для каждого из 4-х выходов:

\lstinputlisting{crypto/1round.txt}

Каким-то образом, выражение для каждого последующего выхода больше. Надеюсь, я нигде не ошибся?
И это просто для одного раунда.
Для двух раундов, размер всех 4-х выражений $\approx 970KB$.
Для трех, это $\approx 115MB$.
Для четырех, у меня не хватило памяти на моем компьютере.
Выражения \textit{взрываются} экспоненциально.
А здесь 19 раундов.
Можете ощутить вес.

Вероятно, вы можете упростить эти выражения: здесь очень много лишних скобок,
но я очень пессимистичен, криптоалгоритмы и создаются таким образом, чтобы не иметь лишних операций.

Чтобы взломать его, вы можете использовать эти выражения как систему уравнений и попытаться решить её при помощи SMT-солвера.
Это называется ``алгебраическая атака''.

Другими словами, теоретически, вы можете построить систему уравнений вроде: $MD5(x)=12341234...$,
но эти выражения настолько огромные, что решить это нельзя.
Да, криптографы прекрасно осведомлены об этом и одна из задач успешного шифра в том, чтобы сделать выражения
настолько большими, насколько это возможно, используя разумное время и размер алгоритма.

Тем не менее, вы можете найти много статей о взломе этих криптосистем учитывая сокращенное количество раундов:
пока вырежение еще не \textit{взорвалось}, иногда это возможно.
Это не применимо затем на практике, но подобный опыт имеет некоторые интересные теоретические результаты.

\subsubsection{Попытки взлома ``серьезных'' шифров}

CryptoMiniSat сам по себе существует для поддержки операции исключающего ИЛИ, которая активно используется
в криптографии.

\begin{itemize}
\item Bitcoin mining with SAT solver: \url{http://jheusser.github.io/2013/02/03/satcoin.html}, \url{https://github.com/msoos/sha256-sat-bitcoin}.

\item \href{http://2015.phdays.ru/program/dev/40400/}{Александр Семенов, попытки взлома A5/1, итд. (на русском)}

\item \href{https://yurichev.com/mirrors/SAT_SMT_crypto/thesis-output.pdf}{Vegard Nossum - SAT-based preimage attacks on SHA-1}

\item \href{https://yurichev.com/mirrors/SAT_SMT_crypto/166.pdf}{Algebraic Attacks on the Crypto-1 Stream Cipher in MiFare Classic and Oyster Cards}

\item \href{https://yurichev.com/mirrors/SAT_SMT_crypto/Attacking-Bivium-Using-SAT-Solvers.pdf}{Attacking Bivium Using SAT Solvers}

\item \href{https://yurichev.com/mirrors/SAT_SMT_crypto/Extending_SAT_2009.pdf}{Extending SAT Solvers to Cryptographic Problems}

\item \href{https://yurichev.com/mirrors/SAT_SMT_crypto/sat-hash.pdf}{Applications of SAT Solvers to Cryptanalysis of Hash Functions}

\item \href{https://yurichev.com/mirrors/SAT_SMT_crypto/slidesC2DES.pdf}{Algebraic-Differential Cryptanalysis of DES}

\end{itemize}

\subsection{Любительская криптография}

Это то, что вы можете найти в серийных номерах, ключах с лицензиями, запаковщиками исполняемых файлов, \ac{CTF},
малварь (зловреды), итд.
Иногда даже в ransomware (но в наше время (2017) редко).

Любительскую криптографию очень часто можно взломать используя SMT-солвер, или даже KLEE.

Любительская криптография обычно основывается не на теории, а на визуальной сложности:
если её создатель получает результаты, которые выглядят достаточно хаотичными, часто, он прекращает разрабатывать его далее.
Это даже не безопасность через запутанность (\textit{security through obscurity}),
а даже безопасность через хаотическую путанницу.
Иногда это называется ``The Fallacy of Complex Manipulation''
(см.также \href{https://tools.ietf.org/html/rfc4086}{RFC4086}).

Разработка своего собственного криптоалгоритма это не такая уж и простая вещь.
Это можно сравнить с разработкой своего собственного \ac{PRNG}.
Даже знаменитый Дональд Кнут в создал свой в 1959, и визуально он был очень сложным,
но как потом выяснилось на практике, у него был очень короткий цикл длиной 3178.
[См.также: The Art of Computer Programming том.II стр.4, (1997).]

Самая первая проблема это создание алгоритма, который может генерировать очень длинные выражения.
Частая ошибка это использование операций вроде исключающего ИЛИ и сдвигов/перестановок, что не очень сильно помогает.
Даже хуже: некоторые люди думают, что применение операции исключающего ИЛИ несколько раз может сделать лучше,
например: $(x \oplus 1234) \oplus 5678$.
Очевидно, эти две операции (точнее, любое их количество) можно сократить до одной.
Та же история с применением операций вроде сложения и вычитания --- они все могут быть сокращены до одной операции.

Настоящие криптоалгоритмы вроде IDEA могут использовать несколько операций из разных групп, как исключающее ИЛИ,
сложение и умножение.
Применение их всех в определенном порядке сделает итоговое выражение несократимым.

Когда я готовил эту часть, я попробовал сделать пример любительской хэш-функции:

\lstinputlisting{crypto/1.c}

KLEE может сломать её без особого труда.
Фцнкции такой сложности часто присутствуют в shareware, которые проверяют лицензионные ключи, итд.

А вот как мы можем сделать работу KLEE труднее используя сдвиги, зависимые от входов,
и это делает количество возможных входных значений намного больше:

\lstinputlisting{crypto/2.c}

Сложение (или, как говорят криптографы, \href{https://yurichev.com/blog/modulo/}{модульное сложение}) может всё усложнить
еще сильнее:

\lstinputlisting{crypto/3.c}

В качестве упражнения, можете попробовать сделать блочный шифр, который KLEE не сможет сломать.
Это очень отрезвляющий опыт.
Но даже если вы и сможете, это не панацея, у криптографов есть еще масса криптоаналитических методов.

Итог: если вы имеете дело с любительской криптографией, вы всегда можете попробовать KLEE и SMT-солвер.
И даже более того: иногда у вас есть только ф-ция для дешифрования, и если алгоритм достаточно прост,
при помощи KLEE или SMT-солвера, можно вернуть всё назад.

Еще одна смешная вещь: если вы пытаетесь реализовать любительский криптоалгоритм на языке Verilog/VHDL чтобы запустить
его на \ac{FPGA}, может быть, с целями брутфорса, вы можете обнаружить, что инструменты \ac{EDA} могут оптимизировать
его во время синтеза (это слово, которое они используют вместо ``компиляция''), и может оставить этот криптоалгоритм
в намного меньшем размере, работающим быстрее, чем это было в начале.
Даже если вы попытаетесь реализовать \textit{в железе} алгоритм DES с фиксированным ключом,
Altera Quartus может оптимизировать его первый раунд, и он будет немного меньше остальных.

\subsubsection{Ошибки}

Еще одна выдающаяся особенность любительской криптографии это ошибки.
Ошибки часто остаются невыявленными, потому что выход ф-ции шифрования визуально выглядит ``достаточно хорошим''
или ``достаточно запутанным'', так что создатель бросает работу над ним.

Это особенно справедливо для хэширующих ф-ций, потому что когда вы работаете над блочным шифром, вам нужны
две ф-ции (шифрование/дешифрование), в то время как хэширующая ф-ция одна.

Самый странный любительский криптоалгоритм виденный мною, шифровал только нечетные байты входного блока,
оставляя четные байты нетронутыми, так что входной текст частично присутствовал в итоговом зашифрованном блоке.
Это была ф-ция шифрования использованная в проверке лицензионного ключа.
Трудно поверить в то, что кто-то сделал это намеренно.
Скорее всего, это была просто незамеченная ошибка.

\subsubsection{XOR-шифры}

Самый простой любительский криптоалгоритм просто применяет исключающее ИЛИ используя некоторую таблицу.
В русском языке также применятся термин ``гаммирование''.
Может быть даже еще проще. Вот реальный алгоритм, который я однажды видел:

\begin{lstlisting}
for (i=0; i<size; i++)
    buf[i]=buf[i]^(31*(i+1));
\end{lstlisting}

Это даже и не шифрование, скорее, сокрытие или упрятывание.

Некоторые другие примеры простейшего криптоанализа XOR-шифров, есть в книге ``Reverse Engineering для начинающих''
\footnote{\url{http://beginners.re}}.

\subsubsection{Другие особенности}

\textbf{Таблицы} Часто приствует таблица/таблицы с псевдослучайными данными, которая/которые хаотично используются.

\textbf{Контрольная сумма} Конечные пользователи имеют склонность изменять коды лицензий, серийные номера, итд,
в надежде, что это как-то повлияет на работу программы.
Так что часто присутствует некоторая контрольная сумма: начиная с простого суммирования и \ac{CRC}.
Это близко к \ac{MAC} в настоящей криптографии (в русском языке применяются термины \textit{имитозащита/имитовставка}).

\textbf{Уровень энтропии} Может быть (намного) ниже, не смотря на то что данные выглядят случайными.

\subsubsection{Примеры}

\begin{itemize}

\item Популярный менеджер лицензий FLEXlm использовал простой любительский криптоалгоритм
(перед тем, как они переключились на \ac{ECC}), который легко можно было сломать.

\item Pegasus Mail Password Decoder: \url{http://phrack.org/issues/52/3.html} -
очень типичный пример.

\item Вы можете найти массу постов в блогах о взломе криптографии уровня \ac{CTF} используя Z3, итд.
Вот один из них: \url{http://doar-e.github.io/blog/2015/08/18/keygenning-with-klee/}.

\item Еще: \href{http://blog.cr4.sh/2015/03/automated-algebraic-cryptanalysis-with.html}{Automated algebraic cryptanalysis with OpenREIL and Z3}.
Кстати, это решение следит за состоянием каждого регистра на каждом EIP/RIP, а это почти то же самое, что и \ac{SSA},
которая активно применяется в компиляторах, и стоит изучения.

\item Массу примеров любительской криптографии я взял со старого сайта Fravia:
\url{https://yurichev.com/mirrors/amateur_crypto_examples_from_Fravia/}.

\item Книга Дмитрия Склярова --- ``Искусство защиты и взлома информации'' имеет много примеров и любительской криптографии,
и использования обычной криптографии с ошибками.

\end{itemize}

% subsection
\subsection{Пример: простая хэш-функция}

(Этот текст был впервые добавлен в мою книгу ``Reverse Engineering для начинающих'' (\url{beginners.re}) в марте 2014
\footnote{Этот пример также использовался Murphy Berzish в его лекции о \ac{SAT} и \ac{SMT}:
\url{http://mirror.csclub.uwaterloo.ca/csclub/mtrberzi-sat-smt-slides.pdf},
\url{http://mirror.csclub.uwaterloo.ca/csclub/mtrberzi-sat-smt.mp4}}.)

Вот необратимая хэш-функция, которая конвертирует одно 64-битное значение в другое,
и нам нужно попытаться развернуть её работу назад.

\subsubsection{Ручная декомпиляция}

Вот листинг на ассемблере в IDA:

\lstinputlisting{crypto/hash/algo_1.asm}

Пример был скомпилирован в GCC, so the first argument is passed in ECX.

Если вы не имеете Hex-Rays, либо вы не доверяете его результатам, мы можем попробовать
переписать всё это на Си вручную.
Один из методов, это представить регистры \ac{CPU} в виде локальных переменных Си и заменить каждую инструкцию
эквивалентным выражением, например:

\lstinputlisting{crypto/hash/algo_2.c}

Если быть достаточно аккуратным, этот код можно скомпилировать, и он даже будет работать, 
точно так же, как оригинальный.

Затем, будем переписывать его постепенно, не забывая об использовании регистров.
Внимание и фокусирование здесь крайне важно --- любая самая мелкая опечатка может испортить всю работу!

Первый шаг:

\lstinputlisting{crypto/hash/algo_3.c}

Следующий шаг:

\lstinputlisting{crypto/hash/algo_4.c}


Мы находим деление через умножение.
Действительно, найдем делитель в Wolfram Mathematica:

\begin{lstlisting}[caption=Wolfram Mathematica]
In[1]:=N[2^(64 + 5)/16^^8888888888888889]
Out[1]:=60.
\end{lstlisting}

Получаем:

\lstinputlisting{crypto/hash/algo_5.c}

Еще один шаг:

\lstinputlisting{crypto/hash/algo_6.c}

Простым сокращением, мы видим, что вычислялось вовсе не частное, а остаток от деления:

\lstinputlisting{crypto/hash/algo_7.c}

Заканчиваем на приятно отформатированном исходном коде:

\lstinputlisting{crypto/hash/algo_src.c}

Так как мы не криптоаналитики, мы не можем найти простой способ найти входное значение
для определенного выходного значения.
Коэффициенты инструкций сдвигов выглядят очень пугающе --- это гарантия что функция не биективная,
она скорее сюръективная,
она имеет коллизии, или, говоря проще, возможны несколько значений на входе для одного на выходе.

Брут-форс это тоже не решение, т.к., значения 64-битные, и это совершенно нереально.

\subsubsection{Попробуем Z3}

Но все же, без всяких специальных знаний из криптографии, мы можем попытаться взломать алгоритм при помощи Z3.

Вот исходный код на Питоне:

\lstinputlisting[numbers=left]{crypto/hash/1.py}

Это будет наш первый солвер.

На строке 7 мы видим объявление переменных.
Это просто 64-битные переменные.
\TT{i1..i6} это промежуточные переменные, отражающие значения в регистрах между исполнениями инструкций.

Потом добавляем т.н. констрайнты, в строках 10..15.
Самый последний констрайнт в строке 17 это наиболее важный: мы будем искать входное значение для
нашего алгоритма, при котором он выдаст на выходе 10816636949158156260.

\textit{RotateRight, RotateLeft, URem} --- это функции из Питоновского Z3 \ac{API} для описания выражений, 
они не связаны с ЯП Python.

Запускаем:

\begin{lstlisting}
...>python.exe 1.py
sat
[i1 = 3959740824832824396,
 i3 = 8957124831728646493,
 i5 = 10816636949158156260,
 inp = 1364123924608584563,
 outp = 10816636949158156260,
 i4 = 14065440378185297801,
 i2 = 4954926323707358301]
 inp=0x12EE577B63E80B73
outp=0x961C69FF0AEFD7E4
\end{lstlisting}

``sat'' означает ``satisfiable'', т.е. солвер нашел по крайней мере одно решение.
Решение выведено внутри квадратных скобок.
Две последние строки это пара входного/выходного значения в шестнадцатеричном виде.
Да, действительно, если мы запустим нашу функцию с
\TT{0x12EE577B63E80B73} на входе, алгоритм выдаст искомое значение.

Но, как мы заметили ранее, функция не биективная, так что тут могут быть и другие корректные входные значения.
Z3 SMT-солвер не выдает результаты больше одного, но мы можем хакнуть наш пример немного, 
добавив констрайнт в строке 19, означая, что мы ищем какие угодно другие результаты кроме этого:

\lstinputlisting[numbers=left]{crypto/hash/2.py}

Действительно, получаем еще один верный результат:

\begin{lstlisting}
...>python.exe 2.py
sat
[i1 = 3959740824832824396,
 i3 = 8957124831728646493,
 i5 = 10816636949158156260,
 inp = 10587495961463360371,
 outp = 10816636949158156260,
 i4 = 14065440378185297801,
 i2 = 4954926323707358301]
 inp=0x92EE577B63E80B73
outp=0x961C69FF0AEFD7E4
\end{lstlisting}

Это можно автоматизировать.
Каждый найденный результат можно добавлять в качестве констрайнта и искать следующий.
Пример немного сложнее:

\lstinputlisting[numbers=left]{crypto/hash/3.py}

Получаем:

\begin{lstlisting}
1364123924608584563
1234567890
9223372038089343698
4611686019661955794
13835058056516731602
3096040143925676201
12319412180780452009
7707726162353064105
16931098199207839913
1906652839273745429
11130024876128521237
15741710894555909141
6518338857701133333
5975809943035972467
15199181979890748275
10587495961463360371
results total= 16
\end{lstlisting}

Так что имеется 16 верных входных значений для \TT{0x92EE577B63E80B73} на выходе.

Второй это 1234567890 --- действительно, это значение было использовано изначально,
при подготовке этого примера.

Попробуем изучить алгоритм немного больше.
В порыве садистских желаний, попробуем найти, есть ли здесь какая-нибудь возможная пара входов/выходов,
в которых младшие 32-битные части равны друг другу?

Уберем констрайнт \textit{outp} и добавим другой, в строке 17:

\lstinputlisting[numbers=left]{crypto/hash/4.py}

И действительно:

\begin{lstlisting}
sat
[i1 = 14869545517796235860,
 i3 = 8388171335828825253,
 i5 = 6918262285561543945,
 inp = 1370377541658871093,
 outp = 14543180351754208565,
 i4 = 10167065714588685486,
 i2 = 5541032613289652645]
 inp=0x13048F1D12C00535
outp=0xC9D3C17A12C00535
\end{lstlisting}

Можем упражняться в садизме и далее: пусть последние 16-бит всегда будут \TT{0x1234}:

\lstinputlisting[numbers=left]{crypto/hash/5.py}

Это так же возможно:

\begin{lstlisting}
sat
[i1 = 2834222860503985872,
 i3 = 2294680776671411152,
 i5 = 17492621421353821227,
 inp = 461881484695179828,
 outp = 419247225543463476,
 i4 = 2294680776671411152,
 i2 = 2834222860503985872]
 inp=0x668EEC35F961234
outp=0x5D177215F961234
\end{lstlisting}

Z3 работает крайне быстро и это означает что алгоритм слаб, и вообще не относится к криптографическим 
(как и почти вся любительская криптография).



