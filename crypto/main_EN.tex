\section{(Amateur) cryptography}
\label{crypto}

\subsection{\textit{Serious} cryptography}

Let's back to the method we previously used (\ref{symbolic_exec}) to construct expressions using running Python function.

We can try to build expression for the output of XXTEA encryption algorithm:

\lstinputlisting{crypto/xxtea.py}

A key is choosen according to input data, and, obviously, we can't know it during symbolic execution, so we leave expression like \TT{k[...]}.

Now results for just one round, for each of 4 outputs:

\lstinputlisting{crypto/1round.txt}

Somehow, size of expression for each subsequent output is bigger. I hope I haven't been mistaken?
And this is just for 1 round.
For 2 rounds, size of all 4 expression is $\approx 970KB$.
For 3 rounds, this is $\approx 115MB$.
For 4 rounds, I have not enough RAM on my computer.
Expressions \textit{exploding} exponentially.
And there are 19 rounds.
You can weigh it.

Perhaps, you can simplify these expressions: there are a lot of excessive parenthesis,
but I'm highly pessimistic, cryptoalgorithms constructed in such a way to not have any spare operations.

In order to crack it, you can use these expressions as system of equation and try to solve it using SMT-solver.
This is called ``algebraic attack''.

In other words, theoretically, you can build system of equation like this: $MD5(x)=12341234...$,
but expressions are so huge so it's impossible to solve them.
Yes, cryptographers are fully aware of this and one of the goals of the successful cipher is
to make expressions as big as possible, using resonable time and size of algorithm.

Nevertheless, you can find numerous papers about breaking these cryptosystems with reduced number of rounds:
when expression isn't \textit{exploded} yet, sometimes it's possible.
This cannot be applied in practice, but such experience has some interesting theoretical results.

\subsubsection{Attempts to break ``serious'' crypto}

CryptoMiniSat itself exist to support XOR operation, which is ubiquitous in cryptography.

\begin{itemize}
\item Bitcoin mining with SAT solver: \url{http://jheusser.github.io/2013/02/03/satcoin.html}, \url{https://github.com/msoos/sha256-sat-bitcoin}.

\item \href{http://2015.phdays.ru/program/dev/40400/}{Alexander Semenov, attempts to break A5/1, etc. (Russian presentation)}

\item \href{https://yurichev.com/mirrors/SAT_SMT_crypto/thesis-output.pdf}{Vegard Nossum - SAT-based preimage attacks on SHA-1}

\item \href{https://yurichev.com/mirrors/SAT_SMT_crypto/166.pdf}{Algebraic Attacks on the Crypto-1 Stream Cipher in MiFare Classic and Oyster Cards}

\item \href{https://yurichev.com/mirrors/SAT_SMT_crypto/Attacking-Bivium-Using-SAT-Solvers.pdf}{Attacking Bivium Using SAT Solvers}

\item \href{https://yurichev.com/mirrors/SAT_SMT_crypto/Extending_SAT_2009.pdf}{Extending SAT Solvers to Cryptographic Problems}

\item \href{https://yurichev.com/mirrors/SAT_SMT_crypto/sat-hash.pdf}{Applications of SAT Solvers to Cryptanalysis of Hash Functions}

\item \href{https://yurichev.com/mirrors/SAT_SMT_crypto/slidesC2DES.pdf}{Algebraic-Differential Cryptanalysis of DES}

\end{itemize}

\subsection{Amateur cryptography}

This is what you can find in serial numbers, license keys, executable file packers, \ac{CTF}, malware, etc.
Sometimes even ransomware (but rarely nowadays, in 2017).

Amateur cryptography is often can be broken using SMT solver, or even KLEE.

Amateur cryptography is usually based not on theory, but on visual complexity:
if its creator getting results which are seems chaotic enough, often, one stops to improve it further.
This is security not even on obscurity, but on chaotic mess.
This is also sometimes called ``The Fallacy of Complex Manipulation''
(see \href{https://tools.ietf.org/html/rfc4086}{RFC4086}).

Devising your own cryptoalgorithm is a very tricky thing to do.
This can be compared to devising your own \ac{PRNG}.
Even famous Donald Knuth in 1959 constructed one, and it was visually very complex,
but, as it turns out in practice, it has very short cycle of length 3178.
[See also: The Art of Computer Programming vol.II page 4, (1997).]

The very first problem is that making an algorithm which can generate very long expressions is tricky thing itself.
Common error is to use operations like XOR and rotations/permutations, which can't help much.
Even worse: some people think that XORing a value several times can be better, like: $(x \oplus 1234) \oplus 5678$.
Obviously, these two XOR operations (or more precisely, any number of it) can be reduced to a single one.
Same story about applied operations like addition and subtraction---they all also can be reduced to single one.

Real cryptoalgorithms, like IDEA, can use several operations from different groups, like XOR, addition and multiplication.
Applying them all in specific order will make resulting expression irreducible.

When I prepared this part, I tried to make an example of such amateur hash function:

\lstinputlisting{crypto/1.c}

KLEE can break it with little effort.
Functions of such complexity is common in shareware, which checks license keys, etc.

Here is how we can make its work harder by making rotations dependent of inputs,
and this makes number of possible inputs much, much bigger:

\lstinputlisting{crypto/2.c}

Addition (or \href{https://yurichev.com/blog/modulo/}{modular addition}, as cryptographers say) can make thing even harder:

\lstinputlisting{crypto/3.c}

As en exercise, you can try to make a block cipher which KLEE wouldn't break.
This is quite sobering experience.
But even if you can, this is not a panacea, there is an array of other cryptoanalytical methods to break it.

Summary: if you deal with amateur cryptography, you can always give KLEE and SMT solver a try.
Even more: sometimes you have only decryption function, and if algorithm is simple enough,
KLEE or SMT solver can reverse things back.

One fun thing to mention: if you try to implement amateur cryptoalgorithm in Verilog/VHDL language to run it on \ac{FPGA},
maybe in brute-force way,
you can find that \ac{EDA} tools can optimize many things during synthesis
(this is the word they use for ``compilation'') and can leave cryptoalgorithm much smaller/simpler than it was.
Even if you try to implement DES algorithm \textit{in bare metal} with a fixed key,
Altera Quartus can optimize first round of it and make it smaller than others.

\subsubsection{Bugs}

Another prominent feature of amateur cryptography is bugs.
Bugs here often left uncaught because output of encrypting function visually looked ``good enough'' or ``obfuscated enough'',
so a developer stopped to work on it.

This is especially feature of hash functions, because when you work on block cipher, you have to do two functions
(encryption/decryption), while hashing function is single.

Weirdest ever amateur encryption algorithm I once saw, encrypted only odd bytes of input block, while even bytes
left untouched, so the input plain text has been partially seen in the resulting encrypted block.
It was encryption routine used in license key validation.
Hard to believe someone did this on purpose.
Most likely, it was just an unnoticed bug.

\subsubsection{XOR ciphers}

Simplest possible amateur cryptography is just application of XOR operation using some kind of table.
Maybe even simpler. This is a real algorithm I once saw:

\begin{lstlisting}
for (i=0; i<size; i++)
    buf[i]=buf[i]^(31*(i+1));
\end{lstlisting}

This is not even encryption, rather concealing or hiding.

\subsubsection{Other features}

\textbf{Tables} There are often table(s) with pseudorandom data, which is/are used chaotically.

\textbf{Checksumming} End-users can have proclivity to changing license codes, serial numbers, etc., with a hope
this could affect behaviour of software.
So there is often some kind of checksum: starting at simple summing and \ac{CRC}.
This is close to \ac{MAC} in real cryptography.

\subsubsection{Examples}

\begin{itemize}

\item A popular FLEXlm license manager was based on a simple amateur cryptoalgorithm
(before they switched to \ac{ECC}), which can be cracked easily.

\item Pegasus Mail Password Decoder: \url{http://phrack.org/issues/52/3.html} -
a very typical example.

\item You can find a lot of blog posts about breaking \ac{CTF}-level crypto using Z3, etc.
Here is one of them: \url{http://doar-e.github.io/blog/2015/08/18/keygenning-with-klee/}.

\item Another: \href{http://blog.cr4.sh/2015/03/automated-algebraic-cryptanalysis-with.html}{Automated algebraic cryptanalysis with OpenREIL and Z3}.
By the way, this solution tracks state of each register at each EIP/RIP,
this is almost the same as \ac{SSA}, which is heavily used in compiers and worth learning.

\item Many examples of amateur cryptography I've taken from an old Fravia website:
\url{https://yurichev.com/mirrors/amateur_crypto_examples_from_Fravia/}.

\end{itemize}

% subsection
\subsection{Case study: simple hash function}

(This piece of text was initially added to my ``Reverse Engineering for Beginners'' book (\url{beginners.re}) at March 2014)
\footnote{This example was also used by Murphy Berzish in his lecture about \ac{SAT} and \ac{SMT}:
\url{http://mirror.csclub.uwaterloo.ca/csclub/mtrberzi-sat-smt-slides.pdf},
\url{http://mirror.csclub.uwaterloo.ca/csclub/mtrberzi-sat-smt.mp4}}.

Here is one-way hash function, that converted a 64-bit value to another and we need to try to reverse its flow back.

\subsubsection{Manual decompiling}

Here its assembly language listing in IDA:

\lstinputlisting{crypto/hash/algo_1.asm}

The example was compiled by GCC, so the first argument is passed in ECX.

If you don't have Hex-Rays, or if you distrust to it, you can try to reverse this code manually.
One method is to represent the CPU registers as local C variables and replace each instruction by
a one-line equivalent expression, like:

\lstinputlisting{crypto/hash/algo_2.c}

If you are careful enough, this code can be compiled and will even work in the same way as the original.

Then, we are going to rewrite it gradually, keeping in mind all registers usage.
Attention and focus is very important here---any tiny typo may ruin all your work!

Here is the first step:

\lstinputlisting{crypto/hash/algo_3.c}

Next step:

\lstinputlisting{crypto/hash/algo_4.c}

We can spot the division using multiplication.
Indeed, let's calculate the divider in Wolfram Mathematica:

\begin{lstlisting}[caption=Wolfram Mathematica]
In[1]:=N[2^(64 + 5)/16^^8888888888888889]
Out[1]:=60.
\end{lstlisting}

We get this:

\lstinputlisting{crypto/hash/algo_5.c}

One more step:

\lstinputlisting{crypto/hash/algo_6.c}

By simple reducing, we finally see that it's calculating the remainder, not the quotient:

\lstinputlisting{crypto/hash/algo_7.c}

We end up with this fancy formatted source-code:

\lstinputlisting{crypto/hash/algo_src.c}

Since we are not cryptoanalysts we can't find an easy way to generate the input value for some specific output value.
The rotate instruction's coefficients look frightening---it's a warranty that the function is not bijective,
it is rather surjective, 
it has collisions, or, speaking more simply, many inputs may be possible for one output.

Brute-force is not solution because values are 64-bit ones, that's beyond reality.

\subsubsection{Now let's use the Z3}

Still, without any special cryptographic knowledge, we may try to break this algorithm using Z3.

Here is the Python source code:

\lstinputlisting[numbers=left]{crypto/hash/1.py}

This is going to be our first solver.

We see the variable definitions on line 7.
These are just 64-bit variables.
\TT{i1..i6} are intermediate variables, representing the values in the registers between instruction executions.

Then we add the so-called constraints on lines 10..15.
The last constraint at 17 is the most important one: 
we are going to try to find an input value for which our algorithm will produce 10816636949158156260.

\textit{RotateRight, RotateLeft, URem}---are functions from the Z3 Python API, not related to Python language.

Then we run it:

\begin{lstlisting}
...>python.exe 1.py
sat
[i1 = 3959740824832824396,
 i3 = 8957124831728646493,
 i5 = 10816636949158156260,
 inp = 1364123924608584563,
 outp = 10816636949158156260,
 i4 = 14065440378185297801,
 i2 = 4954926323707358301]
 inp=0x12EE577B63E80B73
outp=0x961C69FF0AEFD7E4
\end{lstlisting}

``sat'' mean ``satisfiable'', i.e., the solver was able to find at least one solution.
The solution is printed in the square brackets.
The last two lines are the input/output pair in hexadecimal form.
Yes, indeed, if we run our function with \TT{0x12EE577B63E80B73} as input,
the algorithm will produce the value we were looking for.

But, as we noticed before, the function we work with is not bijective, so there may be other correct input values.
The Z3 is not capable of producing more than one result, but let's hack our example slightly, 
by adding line 19, which implies ``look for any other results than this'':

\lstinputlisting[numbers=left]{crypto/hash/2.py}

Indeed, it finds another correct result:

\begin{lstlisting}
...>python.exe 2.py
sat
[i1 = 3959740824832824396,
 i3 = 8957124831728646493,
 i5 = 10816636949158156260,
 inp = 10587495961463360371,
 outp = 10816636949158156260,
 i4 = 14065440378185297801,
 i2 = 4954926323707358301]
 inp=0x92EE577B63E80B73
outp=0x961C69FF0AEFD7E4
\end{lstlisting}

This can be automated.
Each found result can be added as a constraint and then the next result will be searched for.
Here is a slightly more sophisticated example:

\lstinputlisting[numbers=left]{crypto/hash/3.py}

We got:

\begin{lstlisting}
1364123924608584563
1234567890
9223372038089343698
4611686019661955794
13835058056516731602
3096040143925676201
12319412180780452009
7707726162353064105
16931098199207839913
1906652839273745429
11130024876128521237
15741710894555909141
6518338857701133333
5975809943035972467
15199181979890748275
10587495961463360371
results total= 16
\end{lstlisting}

So there are 16 correct input values for \TT{0x92EE577B63E80B73} as a result.

The second is 1234567890---it is indeed the value which was used by me originally while preparing this example.

Let's also try to research our algorithm a bit more.
Acting on a sadistic whim, let's find if there are any possible input/output pairs in 
which the lower 32-bit parts are equal to each other?

Let's remove the \textit{outp} constraint and add another, at line 17:

\lstinputlisting[numbers=left]{crypto/hash/4.py}

It is indeed so:

\begin{lstlisting}
sat
[i1 = 14869545517796235860,
 i3 = 8388171335828825253,
 i5 = 6918262285561543945,
 inp = 1370377541658871093,
 outp = 14543180351754208565,
 i4 = 10167065714588685486,
 i2 = 5541032613289652645]
 inp=0x13048F1D12C00535
outp=0xC9D3C17A12C00535
\end{lstlisting}

Let's be more sadistic and add another constraint: last 16 bits must be \TT{0x1234}:

\lstinputlisting[numbers=left]{crypto/hash/5.py}

Oh yes, this possible as well:

\begin{lstlisting}
sat
[i1 = 2834222860503985872,
 i3 = 2294680776671411152,
 i5 = 17492621421353821227,
 inp = 461881484695179828,
 outp = 419247225543463476,
 i4 = 2294680776671411152,
 i2 = 2834222860503985872]
 inp=0x668EEC35F961234
outp=0x5D177215F961234
\end{lstlisting}

Z3 works very fast and it implies that the algorithm is weak, it is not cryptographic at all
(like the most of the amateur cryptography).



