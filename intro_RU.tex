% (Недобавленный эпиграф):
% Molfara: Чем ты сейчас занимаешься?
% Я: (рассказал)
% Molfara: Автоматические доказыватели теорем... о боже, мой мир уже никогда не будет прежним...
% Я: Нет, не будет.

\section{Это черновик!}

Это еще пока ранний черновик, но всё же может быть кому-то интересен.

Последняя версия всегда доступна на \url{http://yurichev.com/writings/SAT_SMT_draft-RU.pdf}.
Англоязычная версия: \url{http://yurichev.com/writings/SAT_SMT_draft-EN.pdf}.

Версия этого текста: \today{}.

Для получения новостей об обновлениях, можете подписаться на мой
twitter\footnote{\url{https://twitter.com/yurichev}}, 
facebook\footnote{\url{https://www.facebook.com/dennis.yurichev.5}}, 
или репозиторий на github\footnote{\url{https://github.com/dennis714/SAT_SMT_article}}.

\section{Благодарности}

Leonardo de Moura и Nikolaj Bjorner, за помощь.

\section{Введение}

\ac{SAT}/\ac{SMT} солверы можно рассматривать как солверы огромных систем уравнений.
Разница в том, что \ac{SMT}-солверы берут системы в произвольном формате,
в то время как \ac{SAT}-солверы ограничены булевыми уравнениями вида \ac{CNF}.

Огромное количество проблем их практики можно представить как проблемы решения систем уравнений.

\section{Это хайп? Очередная мода?}

Некоторые люди говорят, что это очередной хайп.
Нет, \ac{SAT} достаточно стар, чтобы быть фундаментальным в \ac{CS}.
Причина повышенного интереса в том, что компьютеры стали работать быстрее,
так что теперь больше попыток решать старые проблемы используя 
\ac{SAT}/\ac{SMT}, которые раннее были недоступны.

