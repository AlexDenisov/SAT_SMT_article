\subsection{Менеджер пакетов и Z3}

Вот упрощенный пример.
У нас есть libA, libB, libC и libD, доступные в разных версиях (и видах).
Мы собираемся заинсталлировать programA и programB, которые используют эти библиотеки.

% TODO: translate
\lstinputlisting{SMT/dep/dependency.py}

( Исходный код: \url{https://github.com/dennis714/SAT_SMT_article/blob/master/SMT/dep/dependency.py} )

Вывод:

\begin{lstlisting}
sat
[libB = 5,
 libD = 999,
 libC = 10,
 programB = 7,
 programA = 1,
 libA = 2]
\end{lstlisting}

999 означает, что нет необходимости инсталлировать libD, она не требуется другими пакетами.

Поменяйте версию ProgramB на v8 и оно скажет ``unsat'', означая, что здесь есть конфликт:
ProgramA требует libA v2, но ProgramB v8 в итоге требует более новую libA.

Тут еще есть над чем работать: сообщение ``unsat'' бессмысленно для пользователя, нужно выдать какую-то информацию
о конфликтующих объектах.

Вот еще один мой пример связанный с оптимизационной проблемой: \ref{set_cover}.

Еще об использовании SAT/SMT-солверов в менеджерах пакетов: \url{https://research.swtch.com/version-sat},
\url{https://cseweb.ucsd.edu/~lerner/papers/opium.pdf}.

И теперь в другую сторону: использование менеджера пакетов aptitude для решения Судоку: \\
\url{http://web.archive.org/web/20160326062818/http://algebraicthunk.net/~dburrows/blog/entry/package-management-sudoku/}.

Некоторые читатели могут спросить, как упорядочить библиотеки/программы/пакеты, которые нужно инсталлировать?
Это более простая проблема, часто решаемая топологической сортировкой.
Этот алгоритм упорядочивает граф таким образом, что узлы не зависящие от других будет в самом верху очереди.
Затем, пойдут узлы зависимые от узлов с предыдущего слоя.
И так далее.

Утилита \textit{make} в UNIX делает это во время конструирования порядка объектов для обработки.
И более того: более старые версии утилиты \textit{make} вызывали для этого внешнюю утилиту (\textit{tsort}).
Она есть в более старых UNIX-ах, как минимум в некоторых версиях NetBSD
\footnote{\url{http://netbsd.gw.com/cgi-bin/man-cgi/man?tsort+1+NetBSD-current}}.

