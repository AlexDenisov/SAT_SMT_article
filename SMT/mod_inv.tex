\subsection{Finding modular inverse}

And now one useful equation.

It's possible to replace division operation (slow) by multiplication (faster).
Read \href{http://beginners.re/15-Nov-2015/Reverse_Engineering_for_Beginners-en.pdf#page=490&zoom=auto,-107,806}{here} or \href{http://yurichev.com/blog/modulo/}{here} about it.

In shortest possible terms, here is how it works.

If you need to divide integer by 13, this is the same as multiplication it by $\frac{1}{13}$.
But you operate on integer numbers (\ac{CPU} registers) and can't use ratios.

Given the fact that the division by $2^n$ on \ac{CPU} occurring at lightning speed, we can reformulate the problem as:

\begin{center}
{\Large $\frac{x}{13} = x \cdot \frac{1}{13} = x \cdot \frac{M}{2^n} = \frac{x \cdot M}{2^n}$ }
\end{center}

... where $M$ is \textit{magic coefficient} and $n$ is usually started at register width in bits (32 or 64).

But how to find \textit{magic coefficient}?
This equation is to be solved:

\begin{center}
{\Large $17 \cdot x = 1+k \cdot 2^{64}$ }
\end{center}

This is a little overkill to use SMT-solver to solve this equation
\footnote{\url{https://en.wikipedia.org/wiki/Extended_Euclidean_algorithm} is usually used:
\url{http://rosettacode.org/wiki/Modular\_inverse}}, but still possible:

\begin{lstlisting}
(declare-const x Int)
(declare-const k Int)
(assert (> x 0))
(assert (> k 0))
(assert (=
		(* 17 x)
		(+ 1 (* k (^ 2 64)))
	)
)
(check-sat)
(get-model)
\end{lstlisting}

Important note: we use \textit{Int} sort here, because this is Diophantine equation (i.e., only integer solutions are allowed).
Resulting code will run on integer \ac{CPU} registers, after all.

Z3 gave this:

% FIXME:
\begin{lstlisting}
\$ z3 -smt2 1.smt
sat
(model
  (define-fun k () Int
    16)
  (define-fun x () Int
    17361641481138401521)
)
\end{lstlisting}

Let's check:

% FIXME:
\begin{lstlisting}
#include <stdint.h>
#include <stdio.h>

uint64_t div_by_17 (uint64_t x)
{
	__int128 result = (__int128)x * (__int128)17361641481138401521UL;
	return result >> 64 >> 4;
};

int main()
{
	printf ("%lld\n", 123456789/17);
	printf ("%lld\n", div_by_17(123456789));
};
\end{lstlisting}

... and it works.

GCC doesn't use 128-bit registers, it just exploit the fact that x86 multiplication instruction produces 128-bit result, lower 64-bit part in RAX register,
and higher 64-bit part in RDX register.
In standard C/C++ it's not possible to access higher part, so this hack is helpful.

This is what optimizing GCC 4.8.4 generates:

\begin{lstlisting}
div_by_17:
        movabs  rax, -1085102592571150095
        mul     rdi
        mov     rax, rdx
        sar     rax, 4
        ret
\end{lstlisting}

Z3 may give ``unsat'' (for XXX) % 10?
if there is no integer solution for the equation (but real solution may exist).

