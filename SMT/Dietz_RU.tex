\subsection{Формула Dietz-а}

Еще один впечатляющий пример работы \textit{Aha!} это нахождение формулы Dietz-а\footnote{\url{http://aggregate.org/MAGIC/\#Average\%20of\%20Integers}},
это код для вычисления среднего арифметического из двух чисел без переполнения
(что в свою очередь важно, если вы хотите найти среднее число из таких чисел как 0xFFFFFF00, используя 32-битные регистры).

Используя это на входе:

\begin{lstlisting}
int userfun(int x, int y) {     // To find Dietz's formula for
                                // the floor-average of two
                                // unsigned integers.
   return ((unsigned long long)x + (unsigned long long)y) >> 1;
}
\end{lstlisting}

\dots \textit{Aha!} выдаёт это:

\begin{lstlisting}
Found a 4-operation program:
   and   r1,ry,rx
   xor   r2,ry,rx
   shrs  r3,r2,1
   add   r4,r3,r1
   Expr: (((y ^ x) >>s 1) + (y & x))
\end{lstlisting}

И это работает корректно\footnote{Для тех, кому интересно, как это работает, вся эта механика
очень близка к той, что используется в странной альтернативе XOR, которую мы только что видели.
Вот почему я расположил эти два фрагмента текста один за другим.}.
На как доказать это?

Мы можем поместить формулу Dietz-а в левой стороне уравнения и $x+y/2$ (или $x+y>>1$) в правой стороне:

\begin{center}
$\forall n \in 0..2^{64}-1 . (x\&y) + (x \oplus y)>>1 = x+y>>1$
\end{center}

Одна важная вещь это то что мы не можем оперировать 64-битными значениями в правой части, потому что это приведет
к переполнению.
Так что мы расширяем входы в правой стороне на 1 бит (другими словами, мы просто добавляем 1 нулевой бит перед каждым
значением).
Результат работы формулы Dietz-а также расширяем на 1 бит.
Следовательно, обе части уравнения будут иметь ширину в 65 бит:

\begin{lstlisting}
(declare-const x (_ BitVec 64))
(declare-const y (_ BitVec 64))
(assert 
	(forall ((x (_ BitVec 64)) (y (_ BitVec 64)))
		(=
			((_ zero_extend 1)
				(bvadd
					(bvand x y)
					(bvlshr (bvxor x y) (_ bv1 64))
				)
			)
			(bvlshr
				(bvadd ((_ zero_extend 1) x) ((_ zero_extend 1) y))
				(_ bv1 65)
			)
		)
	)
)
(check-sat)
\end{lstlisting}

Z3 говорит ``sat''.\\
\\
65 достаточно, потому что результат сложения двух самых больших 64-битных значений имеет ширину 65 бит: \\
\TT{0xFF...FF + 0xFF...FF = 0x1FF...FE}.\\
\\
Как и в предыдущем примере с эквивалентом XOR, \TT{(not (= ... ))} и \TT{exists} также можно здесь использовать
вместо \TT{forall}.

