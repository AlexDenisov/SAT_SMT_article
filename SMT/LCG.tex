\subsection{Cracking \ac{LCG} with Z3 \ac{SMT} solver}

This part is first appeared in my blog in June 2015 at \url{http://yurichev.com/blog/modulo/}.

There are well-known weaknesses of LCG (
\href{http://en.wikipedia.org/wiki/Linear_congruential_generator#Advantages_and_disadvantages_of_LCGs}{1},
\href{http://www.reteam.org/papers/e59.pdf}{2},
\href{http://stackoverflow.com/questions/8569113/why-1103515245-is-used-in-rand/8574774#8574774}{3}
), but let's see, if it would be possible to crack it straightforwardly, without any special knowledge.
We would define all relations between LCG states in term of Z3 \ac{SMT} solver.
(I first made attempt to do it using \href{https://reference.wolfram.com/language/ref/FindInstance.html}{FindInstance} in Wolfram Mathematica, but failed, perhaps, made a mistake somewhere).
Here is a test progam:

\begin{lstlisting}
#include <stdlib.h>
#include <stdio.h>
#include <time.h>

int main()
{
	int i;

	srand(time(NULL));

	for (i=0; i<10; i++)
		printf ("%d\n", rand()%100);
};
\end{lstlisting}

It is intended to print 10 pseudorandom numbers in 0..99 range.
So it does:

\begin{lstlisting}
37
29
74
95
98
40
23
58
61
17
\end{lstlisting}

Let's say we are observing only 8 of these numbers (from 29 to 61) and we need to predict next one (17) and/or previous one (37).

The program is compiled using MSVC 2013 (I choose it because its LCG is simpler than that in Glib):

\begin{lstlisting}
.text:0040112E rand            proc near
.text:0040112E                 call    __getptd
.text:00401133                 imul    ecx, [eax+0x14], 214013
.text:0040113A                 add     ecx, 2531011
.text:00401140                 mov     [eax+14h], ecx
.text:00401143                 shr     ecx, 16
.text:00401146                 and     ecx, 7FFFh
.text:0040114C                 mov     eax, ecx
.text:0040114E                 retn
.text:0040114E rand            endp
\end{lstlisting}

This is very simple LCG, but the result is not clipped state, but it's rather shifted by 16 bits.
Let's define LCG in Z3:

\begin{lstlisting}
#!/usr/bin/python
from z3 import *

output_prev = BitVec('output_prev', 32)
state1 = BitVec('state1', 32)
state2 = BitVec('state2', 32)
state3 = BitVec('state3', 32)
state4 = BitVec('state4', 32)
state5 = BitVec('state5', 32)
state6 = BitVec('state6', 32)
state7 = BitVec('state7', 32)
state8 = BitVec('state8', 32)
state9 = BitVec('state9', 32)
state10 = BitVec('state10', 32)
output_next = BitVec('output_next', 32)

s = Solver()

s.add(state2 == state1*214013+2531011)
s.add(state3 == state2*214013+2531011)
s.add(state4 == state3*214013+2531011)
s.add(state5 == state4*214013+2531011)
s.add(state6 == state5*214013+2531011)
s.add(state7 == state6*214013+2531011)
s.add(state8 == state7*214013+2531011)
s.add(state9 == state8*214013+2531011)
s.add(state10 == state9*214013+2531011)

s.add(output_prev==URem((state1>>16)&0x7FFF,100))
s.add(URem((state2>>16)&0x7FFF,100)==29)
s.add(URem((state3>>16)&0x7FFF,100)==74)
s.add(URem((state4>>16)&0x7FFF,100)==95)
s.add(URem((state5>>16)&0x7FFF,100)==98)
s.add(URem((state6>>16)&0x7FFF,100)==40)
s.add(URem((state7>>16)&0x7FFF,100)==23)
s.add(URem((state8>>16)&0x7FFF,100)==58)
s.add(URem((state9>>16)&0x7FFF,100)==61)
s.add(output_next==URem((state10>>16)&0x7FFF,100))

print(s.check())
print(s.model())
\end{lstlisting}

URem states for \textit{unsigned remainder}.
It works for some time and gave us correct result!

\begin{lstlisting}
sat
[state3 = 2276903645,
 state4 = 1467740716,
 state5 = 3163191359,
 state7 = 4108542129,
 state8 = 2839445680,
 state2 = 998088354,
 state6 = 4214551046,
 state1 = 1791599627,
 state9 = 548002995,
 output_next = 17,
 output_prev = 37,
 state10 = 1390515370]
\end{lstlisting}

% FIXME tilde
I added ~10 states to be sure result will be correct. It may be not if you supply lesser amount of PRNG numbers.

That is the reason why LCG is not suitable for any security-related task.
This is why \href{https://en.wikipedia.org/wiki/Cryptographically_secure_pseudorandom_number_generator}{cryptographically secure pseudorandom number generators} exist: they are designed to be protected against such simple attack.
Well, at least if \href{https://en.wikipedia.org/wiki/Dual_EC_DRBG}{NSA is not involved}.

As far, as I can understand, \href{http://en.wikipedia.org/wiki/Security_token}{security tokens} like \href{http://en.wikipedia.org/wiki/RSA_SecurID}{RSA SecurID} can be viewed just as \ac{CPRNG} with a secret seed.
It shows new pseudorandom number each minute, and the server can predict it, because it knows the seed.
Imagine if such token would implement LCG -- it would be much easier to break!

