\section{Alphametics}

According to Donald Knuth, the term ``Alphametics'' was coined by J. A. H. Hunter.
This is a puzzle: what decimal digits in 0..9 range must be assigned to each letter,
so the following equation will be true?

\begin{lstlisting}
  SEND
+ MORE
 -----
 MONEY
\end{lstlisting}

This is easy for Z3:

\lstinputlisting{SMT/alpha.py}

Output:

\begin{lstlisting}
sat
[E, = 5,
 S, = 9,
 M, = 1,
 N, = 6,
 D, = 7,
 R, = 8,
 O, = 0,
 Y = 2]
\end{lstlisting}

Another one, also from \ac{TAOCP} volume IV (\url{http://www-cs-faculty.stanford.edu/~uno/fasc2b.ps.gz}):

\lstinputlisting{SMT/alpha2.py}

\begin{lstlisting}
sat
[L, = 6,
 S, = 7,
 N, = 2,
 T, = 1,
 I, = 5,
 V = 3,
 A, = 8,
 R, = 9,
 O, = 4,
 TRIO = 1954,
 SONATA, = 742818,
 VIOLA, = 35468,
 VIOLIN, = 354652]
\end{lstlisting}

This puzzle I've found in pySMT examples:

\lstinputlisting{SMT/alpha3.py}

\begin{lstlisting}
sat
[E, = 4, D = 6, O, = 2, W, = 3, R, = 5, L, = 1, H, = 9]
\end{lstlisting}

