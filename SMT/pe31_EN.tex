\subsection{Solving Problem Euler 31: ``Coin sums''}

(This text was first published in my blog\footnote{\url{http://dennisyurichev.blogspot.de/2013/05/in-england-currency-is-made-up-of-pound.html}} at 10-May-2013.)

\begin{framed}
\begin{quotation}
In England the currency is made up of pound, £, and pence, p, and there are eight coins in general circulation:

1p, 2p, 5p, 10p, 20p, 50p, £1 (100p) and £2 (200p).
It is possible to make £2 in the following way:

1£1 + 150p + 220p + 15p + 12p + 31p
How many different ways can £2 be made using any number of coins?
\end{quotation}
\end{framed}
( \href{http://projecteuler.net/problem=31}{Problem Euler 31 --- Coin sums} )

\label{SMTEnumerate}
Using Z3 for solving this is overkill, and also slow, but nevertheless, it works, showing all possible solutions as well.
The piece of code for blocking already found solution and search for next, and thus, counting all solutions, was taken from Stack Overflow answer
\footnote{\url{http://stackoverflow.com/questions/11867611/z3py-checking-all-solutions-for-equation}, 
another question: \url{http://stackoverflow.com/questions/13395391/z3-finding-all-satisfying-models}}.
This is also called ``model counting''.
Constraints like ``a>=0'' must be present, because Z3 solver will find solutions with negative numbers.

\begin{lstlisting}
#!/usr/bin/python

from z3 import *

a,b,c,d,e,f,g,h = Ints('a b c d e f g h')
s = Solver()
s.add(1*a + 2*b + 5*c + 10*d + 20*e + 50*f + 100*g + 200*h == 200, 
   a>=0, b>=0, c>=0, d>=0, e>=0, f>=0, g>=0, h>=0)
result=[]

while True:
    if s.check() == sat:
        m = s.model()
        print m
        result.append(m)
        # Create a new constraint the blocks the current model
        block = []
        for d in m:
            # d is a declaration
            if d.arity() > 0:
                raise Z3Exception("uninterpreted functions are not suppported")
            # create a constant from declaration
            c=d()
            #print c, m[d]
            if is_array(c) or c.sort().kind() == Z3_UNINTERPRETED_SORT:
                raise Z3Exception("arrays and uninterpreted sorts are not supported")
            block.append(c != m[d])
        #print "new constraint:",block
        s.add(Or(block))
    else:
        print len(result)
        break
\end{lstlisting}

Works very slow, and this is what it produces:

\begin{lstlisting}
[h = 0, g = 0, f = 0, e = 0, d = 0, c = 0, b = 0, a = 200]
[f = 1, b = 5, a = 0, d = 1, g = 1, h = 0, c = 2, e = 1]
[f = 0, b = 1, a = 153, d = 0, g = 0, h = 0, c = 1, e = 2]
...
[f = 0, b = 31, a = 33, d = 2, g = 0, h = 0, c = 17, e = 0]
[f = 0, b = 30, a = 35, d = 2, g = 0, h = 0, c = 17, e = 0]
[f = 0, b = 5, a = 50, d = 2, g = 0, h = 0, c = 24, e = 0]
\end{lstlisting}

73682 results in total.
