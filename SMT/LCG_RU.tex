\subsection{Взлом \ac{LCG} при помощи Z3}

(Этот текст впервые был опубликован в моем блоге в июне 2015 на \url{http://yurichev.com/blog/modulo/}.)

У \ac{LCG} есть хорошо известные слабости
\footnote{\url{http://en.wikipedia.org/wiki/Linear_congruential_generator\#Advantages_and_disadvantages_of_LCGs},
\url{http://www.reteam.org/papers/e59.pdf},
\url{http://stackoverflow.com/questions/8569113/why-1103515245-is-used-in-rand/8574774\#8574774}},
но посмотрим, можно ли его взломать напрямую, без специальных знаний.
Мы определим все связи между состояниями LCG в терминах Z3.
Вот тестовая программа:

\begin{lstlisting}
#include <stdlib.h>
#include <stdio.h>
#include <time.h>

int main()
{
	int i;

	srand(time(NULL));

	for (i=0; i<10; i++)
		printf ("%d\n", rand()%100);
};
\end{lstlisting}

Она выдает 10 псевдослучайных чисел в пределах 0..99:

\begin{lstlisting}
37
29
74
95
98
40
23
58
61
17
\end{lstlisting}

Скажем, мы наблюдаем только 8 из этих чисел (от 29 до 61) и нам нужно предсказать следующее (17) и/или предыдущее (37).

Программа скомпилирована в MSVC 2013 (я выбрал его потому что тут LCG проще чем в Glib):

\begin{lstlisting}
.text:0040112E rand            proc near
.text:0040112E                 call    __getptd
.text:00401133                 imul    ecx, [eax+0x14], 214013
.text:0040113A                 add     ecx, 2531011
.text:00401140                 mov     [eax+14h], ecx
.text:00401143                 shr     ecx, 16
.text:00401146                 and     ecx, 7FFFh
.text:0040114C                 mov     eax, ecx
.text:0040114E                 retn
.text:0040114E rand            endp
\end{lstlisting}

Определим \ac{LCG} в Z3Py:

\begin{lstlisting}
#!/usr/bin/python
from z3 import *

output_prev = BitVec('output_prev', 32)
state1 = BitVec('state1', 32)
state2 = BitVec('state2', 32)
state3 = BitVec('state3', 32)
state4 = BitVec('state4', 32)
state5 = BitVec('state5', 32)
state6 = BitVec('state6', 32)
state7 = BitVec('state7', 32)
state8 = BitVec('state8', 32)
state9 = BitVec('state9', 32)
state10 = BitVec('state10', 32)
output_next = BitVec('output_next', 32)

s = Solver()

s.add(state2 == state1*214013+2531011)
s.add(state3 == state2*214013+2531011)
s.add(state4 == state3*214013+2531011)
s.add(state5 == state4*214013+2531011)
s.add(state6 == state5*214013+2531011)
s.add(state7 == state6*214013+2531011)
s.add(state8 == state7*214013+2531011)
s.add(state9 == state8*214013+2531011)
s.add(state10 == state9*214013+2531011)

s.add(output_prev==URem((state1>>16)&0x7FFF,100))
s.add(URem((state2>>16)&0x7FFF,100)==29)
s.add(URem((state3>>16)&0x7FFF,100)==74)
s.add(URem((state4>>16)&0x7FFF,100)==95)
s.add(URem((state5>>16)&0x7FFF,100)==98)
s.add(URem((state6>>16)&0x7FFF,100)==40)
s.add(URem((state7>>16)&0x7FFF,100)==23)
s.add(URem((state8>>16)&0x7FFF,100)==58)
s.add(URem((state9>>16)&0x7FFF,100)==61)
s.add(output_next==URem((state10>>16)&0x7FFF,100))

print(s.check())
print(s.model())
\end{lstlisting}

\textit{URem} означает \textit{unsigned remainder} (беззнаковый остаток от деления).
Какое-то время работает и выдает корректный результат!

\begin{lstlisting}
sat
[state3 = 2276903645,
 state4 = 1467740716,
 state5 = 3163191359,
 state7 = 4108542129,
 state8 = 2839445680,
 state2 = 998088354,
 state6 = 4214551046,
 state1 = 1791599627,
 state9 = 548002995,
 output_next = 17,
 output_prev = 37,
 state10 = 1390515370]
\end{lstlisting}

Я добавил $\approx 10$ состояний, чтобы быть уверенным, что результат будет верным.
Но так могло и не быть, в случае с меньшим количеством информации.

Это причина, по которой \ac{LCG} не предназначена для любых задач связанных с безопасностью.
Вот почему существуют криптостойкие генераторы псевдослучайных чисел:
они разработаны с защитой от такой простой атаки.
Ну, по крайней мере, если \ac{NSA} не вмешивается
\footnote{\url{https://en.wikipedia.org/wiki/Dual_EC_DRBG}}.

Токены вроде ``RSA SecurID'' можно рассматривать просто как \ac{CPRNG} с секретным изначальным состоянием (seed).
Они показыают псевдослучайное число каждую минуту, и сервер может его предсказать, потому что знает seed.
Представьте, если бы такой токен использовал \ac{LCG} --- взломать его было бы куда проще!

