\subsection{Unit test: HTML/CSS color}

The most popular ways to represent HTML/CSS color is by English name (e.g., ``red'') and by 6-digit hexadecimal number (e.g., ``\#0077CC'').
There is third, less popular way: if each byte in hexadecimal number has two doubling digits, it can be \textit{abbreviated}, thus, 
``\#0077CC'' can be written just as ``\#07C''.

Let's write a function to convert 3 color components into name (if possible, first priority), 3-digit hexadecimal form (if possible, second priority),
or as 6-digit hexadecimal form (as a last resort).

\lstinputlisting{KLEE/color.c}

There are 5 possible paths in function, and let's see, if KLEE could find them all?
It's indeed so:

% FIXME:
\begin{lstlisting}
% clang -emit-llvm -c -g color.c

% klee color.bc
KLEE: output directory is "/home/klee/klee-out-134"
KLEE: WARNING: undefined reference to function: sprintf
KLEE: WARNING: undefined reference to function: strcpy
KLEE: WARNING ONCE: calling external: strcpy(51867584, 51598960)
KLEE: ERROR: /home/klee/color.c:33: external call with symbolic argument: sprintf
KLEE: NOTE: now ignoring this error at this location
KLEE: ERROR: /home/klee/color.c:28: external call with symbolic argument: sprintf
KLEE: NOTE: now ignoring this error at this location

KLEE: done: total instructions = 479
KLEE: done: completed paths = 19
KLEE: done: generated tests = 5
\end{lstlisting}

We can ignore calls to strcpy() and sprintf(), because we are not really interesting in state of \TT{out} variable.

So there are exactly 5 paths:

% FIXME:
\begin{lstlisting}
% ls klee-last
assembly.ll   run.stats            test000003.ktest     test000005.ktest
info          test000001.ktest     test000003.pc        test000005.pc
messages.txt  test000002.ktest     test000004.ktest     warnings.txt
run.istats    test000003.exec.err  test000005.exec.err
\end{lstlisting}

1st set of input variables will result in ``red'' string:

% FIXME:
\begin{lstlisting}
% ktest-tool --write-ints klee-last/test000001.ktest
ktest file : 'klee-last/test000001.ktest'
args       : ['color.bc']
num objects: 3
object    0: name: b'R'
object    0: size: 1
object    0: data: b'\xff'
object    1: name: b'G'
object    1: size: 1
object    1: data: b'\x00'
object    2: name: b'B'
object    2: size: 1
object    2: data: b'\x00'
\end{lstlisting}

2nd set of input variables will result in ``green'' string:

% FIXME:
\begin{lstlisting}
% ktest-tool --write-ints klee-last/test000002.ktest
ktest file : 'klee-last/test000002.ktest'
args       : ['color.bc']
num objects: 3
object    0: name: b'R'
object    0: size: 1
object    0: data: b'\x00'
object    1: name: b'G'
object    1: size: 1
object    1: data: b'\xff'
object    2: name: b'B'
object    2: size: 1
object    2: data: b'\x00'
\end{lstlisting}

3rd set of input variables will result in ``\#010000'' string:

% FIXME:
\begin{lstlisting}
% ktest-tool --write-ints klee-last/test000003.ktest
ktest file : 'klee-last/test000003.ktest'
args       : ['color.bc']
num objects: 3
object    0: name: b'R'
object    0: size: 1
object    0: data: b'\x01'
object    1: name: b'G'
object    1: size: 1
object    1: data: b'\x00'
object    2: name: b'B'
object    2: size: 1
object    2: data: b'\x00'
\end{lstlisting}

4th set of input variables will result in ``blue'' string:

% FIXME:
\begin{lstlisting}
% ktest-tool --write-ints klee-last/test000004.ktest
ktest file : 'klee-last/test000004.ktest'
args       : ['color.bc']
num objects: 3
object    0: name: b'R'
object    0: size: 1
object    0: data: b'\x00'
object    1: name: b'G'
object    1: size: 1
object    1: data: b'\x00'
object    2: name: b'B'
object    2: size: 1
object    2: data: b'\xff'
\end{lstlisting}

5th set of input variables will result in ``\#F01'' string:

% FIXME:
\begin{lstlisting}
% ktest-tool --write-ints klee-last/test000005.ktest
ktest file : 'klee-last/test000005.ktest'
args       : ['color.bc']
num objects: 3
object    0: name: b'R'
object    0: size: 1
object    0: data: b'\xff'
object    1: name: b'G'
object    1: size: 1
object    1: data: b'\x00'
object    2: name: b'B'
object    2: size: 1
object    2: data: b'\x11'
\end{lstlisting}

These 5 sets of input variables can form a unit test for our function.

