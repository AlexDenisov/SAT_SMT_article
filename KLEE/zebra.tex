\subsection{Zebra puzzle}

Let's revisit zebra puzzle from (\ref{xebra_SMT}).

We just define all variables and add constraints:

\lstinputlisting{KLEE/klee_zebra1.c}

I force KLEE to find distinct values for colors, nationalities, cigarettes, etc, in the same way as I did for Sudoku earlier 
(\ref{sudoku_SMT}).

Let's run it:

% FIXME:
\begin{lstlisting}
% clang -emit-llvm -c -g klee_zebra1.c
...

% klee klee_zebra1.bc
KLEE: output directory is "/home/klee/klee-out-97"
KLEE: WARNING: undefined reference to function: klee_assert
KLEE: WARNING ONCE: calling external: klee_assert(0)
KLEE: ERROR: /home/klee/klee_zebra1.c:130: failed external call: klee_assert
KLEE: NOTE: now ignoring this error at this location

KLEE: done: total instructions = 761
KLEE: done: completed paths = 55
KLEE: done: generated tests = 55
\end{lstlisting}

It works for ~7 seconds on my Intel Core i3-3110M 2.4GHz notebook.
Let's find out path, where \TT{klee\_assert()} has been executed:

% FIXME:
\begin{lstlisting}
% ls klee-last | grep err
test000051.external.err

% ktest-tool --write-ints klee-last/test000051.ktest | less

ktest file : 'klee-last/test000051.ktest'
args       : ['klee_zebra1.bc']
num objects: 25
object    0: name: b'Yellow'
object    0: size: 4
object    0: data: 1
object    1: name: b'Blue'
object    1: size: 4
object    1: data: 2
object    2: name: b'Red'
object    2: size: 4
object    2: data: 3
object    3: name: b'Ivory'
object    3: size: 4
object    3: data: 4

...

object   21: name: b'Horse'
object   21: size: 4
object   21: data: 2
object   22: name: b'Snails'
object   22: size: 4
object   22: data: 3
object   23: name: b'Dog'
object   23: size: 4
object   23: data: 4
object   24: name: b'Zebra'
object   24: size: 4
object   24: data: 5
\end{lstlisting}

This is indeed correct solution.

\TT{klee\_assume()} also can be used this time:

\lstinputlisting{KLEE/klee_zebra2.c}

\dots and this version works slightly faster ($\approx 5$ seconds),
maybe because KLEE is aware of this intrinsic and handles it in a special way?

