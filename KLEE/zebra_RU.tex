\subsection{Головоломка зебры}

Снова вернемся к головоломке зебры (\ref{xebra_SMT}).

Мы просто определяем все переменные и добавляем констрайнты:

\lstinputlisting{KLEE/klee_zebra1.c}

Я заставил KLEE находить отличные друг от друга значения для цветов, национальностей, сигарет, итд, точно также,
как я раннее сделал это для Судоку: (\ref{sudoku_SMT}).

Запускаем:

\begin{lstlisting}
% clang -emit-llvm -c -g klee_zebra1.c
...

% klee klee_zebra1.bc
KLEE: output directory is "/home/klee/klee-out-97"
KLEE: WARNING: undefined reference to function: klee_assert
KLEE: WARNING ONCE: calling external: klee_assert(0)
KLEE: ERROR: /home/klee/klee_zebra1.c:130: failed external call: klee_assert
KLEE: NOTE: now ignoring this error at this location

KLEE: done: total instructions = 761
KLEE: done: completed paths = 55
KLEE: done: generated tests = 55
\end{lstlisting}

Работает $\approx 7$ секунд на моем ноутбуке с Intel Core i3-3110M 2.4GHz.
Найдем путь, где был исполнен \TT{klee\_assert()}:

\begin{lstlisting}
% ls klee-last | grep err
test000051.external.err

% ktest-tool --write-ints klee-last/test000051.ktest | less

ktest file : 'klee-last/test000051.ktest'
args       : ['klee_zebra1.bc']
num objects: 25
object    0: name: b'Yellow'
object    0: size: 4
object    0: data: 1
object    1: name: b'Blue'
object    1: size: 4
object    1: data: 2
object    2: name: b'Red'
object    2: size: 4
object    2: data: 3
object    3: name: b'Ivory'
object    3: size: 4
object    3: data: 4

...

object   21: name: b'Horse'
object   21: size: 4
object   21: data: 2
object   22: name: b'Snails'
object   22: size: 4
object   22: data: 3
object   23: name: b'Dog'
object   23: size: 4
object   23: data: 4
object   24: name: b'Zebra'
object   24: size: 4
object   24: data: 5
\end{lstlisting}

Это действительно корректное решение.

В этот раз можно также использовать \TT{klee\_assume()}:

\lstinputlisting{KLEE/klee_zebra2.c}

\dots и эта версия работает немного быстрее ($\approx 5$ секунд),
может быть потому что KLEE знает об этой \textit{intrinsic} и обращается с ним особым образом?

