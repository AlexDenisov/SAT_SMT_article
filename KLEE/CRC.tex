\subsection{CRC} % FIXME full name

\subsubsection{Buffer alteration case \#1}

Sometimes, you need to alter a piece of data which is \textit{protected} by some kind of checksum or \ac{CRC}, and you can't change checksum or CRC value, but can alter piece of data so that checksum will remain the same.

Let's pretend, we've got a piece of data with ``Hello, world!'' string at the beginning and ``and goodbye'' string at the end.
We can alter 14 characters at the middle, but for some reason, they must be in \textit{a..z} limits, but we can put any characters there.
CRC64 of the whole block must be \TT{0x12345678abcdef12}.

Let's see\footnote{There are several slightly different CRC64 implementations, the one I use here can also be different from popular ones.}:

\lstinputlisting{KLEE/klee_CRC64.c}

Since our code uses memcmp() standard C/C++ function, we need to add \TT{--libc=uclibc} switch, so KLEE will use its own uClibc % FIXME check spelling
implementation. % \ref{} -> closed programs

% FIXME:
\begin{lstlisting}
\$ clang -emit-llvm -c -g klee_CRC64.c

\$ time klee --libc=uclibc klee_CRC64.bc
\end{lstlisting}

It takes about 1 minute (on my Intel Core i3-3110M 2.4GHz notebook) and we getting this:

% FIXME:
\begin{lstlisting}
...
real    0m52.643s
user    0m51.232s
sys     0m0.239s
...
\$ ls klee-last | grep err
test000001.user.err
test000002.user.err
test000003.user.err
test000004.external.err

\$ ktest-tool --write-ints klee-last/test000004.ktest
ktest file : 'klee-last/test000004.ktest'
args       : ['klee_CRC64.bc']
num objects: 1
object    0: name: b'buf'
object    0: size: 46
object    0: data: b'Hello, world!.. qqlicayzceamyw ... and goodbye'
\end{lstlisting}

Maybe it's slow, but definitely faster than bruteforce.
Indeed, $log_2{26^{14}} \approx 65.8$
which is close to 64 bits.
In other words, one need $\approx 14$ latin characters to encode 64 bits.
And KLEE + \ac{SMT} solver needs 64 bits at some place it can alter to make final CRC64 value equal to what we defined.

I tried to reduce length of the \textit{middle block} to 13 characters: no luck for KLEE then, it has no space enough.

\subsubsection{Buffer alteration case \#2}

I went sadistic: what if the buffer must contain the CRC64 value which, after calculation of CRC64, will result in the same value?
Fascinately, % FIXME check spelling
KLEE can solve this.
The buffer will have the following format:

% FIXME:
\begin{lstlisting}
Hello, world! <8-bytes (64-bit value)> and goodbye <6 more bytes>
\end{lstlisting}

% FIXME:
\begin{lstlisting}
int main()
{
#define HEAD_STR "Hello, world!.. "
#define HEAD_SIZE strlen(HEAD_STR)
#define TAIL_STR " ... and goodbye"
#define TAIL_SIZE strlen(TAIL_STR)
// 8 bytes for 64-bit value:
#define MID_SIZE 8
#define BUF_SIZE HEAD_SIZE+TAIL_SIZE+MID_SIZE+6

	char buf[BUF_SIZE];
  
	klee_make_symbolic(buf, sizeof buf, "buf");

	klee_assume (memcmp (buf, HEAD_STR, HEAD_SIZE)==0);

	klee_assume (memcmp (buf+HEAD_SIZE+MID_SIZE, TAIL_STR, TAIL_SIZE)==0);
	
	uint64_t mid_value=*(uint64_t*)(buf+HEAD_SIZE);
	klee_assume (crc64 (0, buf, BUF_SIZE)==mid_value);

	klee_assert(0);

	return 0;
}
\end{lstlisting}

It works:

% FIXME:
\begin{lstlisting}
\$ time klee --libc=uclibc klee_CRC64.bc
...
real    5m17.081s
user    5m17.014s
sys     0m0.319s

\$ ls klee-last | grep err
test000001.user.err
test000002.user.err
test000003.external.err

\$ ktest-tool --write-ints klee-last/test000003.ktest
ktest file : 'klee-last/test000003.ktest'
args       : ['klee_CRC64.bc']
num objects: 1
object    0: name: b'buf'
object    0: size: 46
object    0: data: b'Hello, world!.. T+]\xb9A\x08\x0fq ... and goodbye\xb6\x8f\x9c\xd8\xc5\x00'
\end{lstlisting}

8 bytes between two strings is 64-bit value which equals to CRC64 of this whole block.
Again, it's faster than brute-force way to find it.
If to decrease last spare 6-byte buffer to 4 bytes or less, KLEE works so long so I've stopped it.

\subsubsection{Recovering input data for given CRC32 value of it}

I've always wanted to do so, but everyone knows this is impossible for input buffers larger than 4 bytes.
As my experiments show, it's still possible for tiny input buffers of data, constrained in some way.

The CRC32 value of 6-byte ``SILVER'' string is known: \TT{0xDFA3DFDD}.
KLEE can find this 6-byte string, if it knows that each byte of input buffer is in \textit{A..Z} limits:

\lstinputlisting[numbers=left]{KLEE/klee_SILVER.c}

% FIXME:
\begin{lstlisting}
\$ clang -emit-llvm -c -g klee_SILVER.c
...

\$ klee klee_SILVER.bc
...

\$ ls klee-last | grep err
test000013.external.err

\$ ktest-tool --write-ints klee-last/test000013.ktest
ktest file : 'klee-last/test000013.ktest'
args       : ['klee_SILVER.bc']
num objects: 1
object    0: name: b'str'
object    0: size: 6
object    0: data: b'SILVER'
\end{lstlisting}

Still, it's no magic: if to remove condition at lines 23..25 (i.e., if to relax constraints),
KLEE will produce some other string, which will be still correct for the CRC32 value given.

It works, because 6 Latin characters in \textit{A..Z} limits contain $\approx 28.2$ bits:
$log_2{26^6} \approx 28.2$, which is even smaller value than 32.
In other words, the final CRC32 value holds enough bits to recover $\approx 28.2$ bits of input.

The input buffer can be even bigger, if each byte of it will be in even tighter % FIXME spelling
constraints (decimal digits, binary digits, etc).

\subsubsection{In comparison with other hashing algorithms}

Things are that easy for some other hashing algorithms like \textit{fletcher checksum}, % FIXME URL
but not for cryptographically secure ones (like MD5, SHA1, etc), they are protected from such simple cryptoanalysis. % FIXME \ref{} -> am.crypto

