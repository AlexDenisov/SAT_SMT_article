\subsection{Unit test: strcmp() function}

The standard \TT{strcmp()} function from C library can return 0, -1 or 1, depending of comparison result.

Here is my own implementation of \TT{strcmp()}:

\lstinputlisting{KLEE/strcmp.c}

Let's find out, if KLEE is capable of finding all three paths?
I intentionaly made things simpler for KLEE by limiting input arrays to two 2 bytes or to 1 character + terminal zero byte.

\begin{lstlisting}
% clang -emit-llvm -c -g strcmp.c

% klee strcmp.bc
KLEE: output directory is "/home/klee/klee-out-131"
KLEE: ERROR: /home/klee/strcmp.c:35: invalid klee_assume call (provably false)
KLEE: NOTE: now ignoring this error at this location
KLEE: ERROR: /home/klee/strcmp.c:36: invalid klee_assume call (provably false)
KLEE: NOTE: now ignoring this error at this location

KLEE: done: total instructions = 137
KLEE: done: completed paths = 5
KLEE: done: generated tests = 5

% ls klee-last
assembly.ll   run.stats            test000002.ktest     test000004.ktest
info          test000001.ktest     test000002.pc        test000005.ktest
messages.txt  test000001.pc        test000002.user.err  warnings.txt
run.istats    test000001.user.err  test000003.ktest
\end{lstlisting}

The first two errors are about \TT{klee\_assume()}.
These are input values on which \TT{klee\_assume()} calls are stuck.
We can ignore them, or take a peek out of curiosity:

\begin{lstlisting}
% ktest-tool --write-ints klee-last/test000001.ktest
ktest file : 'klee-last/test000001.ktest'
args       : ['strcmp.bc']
num objects: 2
object    0: name: b'input1'
object    0: size: 2
object    0: data: b'\x00\x00'
object    1: name: b'input2'
object    1: size: 2
object    1: data: b'\x00\x00'

% ktest-tool --write-ints klee-last/test000002.ktest
ktest file : 'klee-last/test000002.ktest'
args       : ['strcmp.bc']
num objects: 2
object    0: name: b'input1'
object    0: size: 2
object    0: data: b'a\xff'
object    1: name: b'input2'
object    1: size: 2
object    1: data: b'\x00\x00'
\end{lstlisting}

Three rest files are the input values for each path inside of my implementation of \TT{strcmp()}:

\begin{lstlisting}
% ktest-tool --write-ints klee-last/test000003.ktest
ktest file : 'klee-last/test000003.ktest'
args       : ['strcmp.bc']
num objects: 2
object    0: name: b'input1'
object    0: size: 2
object    0: data: b'b\x00'
object    1: name: b'input2'
object    1: size: 2
object    1: data: b'c\x00'

% ktest-tool --write-ints klee-last/test000004.ktest
ktest file : 'klee-last/test000004.ktest'
args       : ['strcmp.bc']
num objects: 2
object    0: name: b'input1'
object    0: size: 2
object    0: data: b'c\x00'
object    1: name: b'input2'
object    1: size: 2
object    1: data: b'a\x00'

% ktest-tool --write-ints klee-last/test000005.ktest
ktest file : 'klee-last/test000005.ktest'
args       : ['strcmp.bc']
num objects: 2
object    0: name: b'input1'
object    0: size: 2
object    0: data: b'a\x00'
object    1: name: b'input2'
object    1: size: 2
object    1: data: b'a\x00'
\end{lstlisting}

3rd is about first argument (``b'') is lesser than the second (``c'').
4th is opposite (``c'' and ``a'').
5th is when they are equal (``a'' and ``a'').

Using these 3 test cases, we've got full coverage of our implementation of \TT{strcmp()}.

