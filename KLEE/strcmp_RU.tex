\subsection{Unit-тест: ф-ция strcmp()}

Стандартная ф-ция Си \TT{strcmp()} может возвращать 0, -1 или 1, в зависимости от результата сравнения.

Вот моя собственная реализация \TT{strcmp()}:

\lstinputlisting{KLEE/strcmp.c}

Попробуем узнать, способен ли KLEE найти все три пути?
Я специально сделал всё как можно проще для KLEE, ограничив входные массивы до двух байт, или до одного байта +
оконечивающий нулевой байт.

\begin{lstlisting}[basicstyle=\ttfamily, mathescape]
% clang -emit-llvm -c -g strcmp.c

% klee strcmp.bc
KLEE: output directory is "/home/klee/klee-out-131"
KLEE: ERROR: /home/klee/strcmp.c:35: invalid klee_assume call (provably false)
KLEE: NOTE: now ignoring this error at this location
KLEE: ERROR: /home/klee/strcmp.c:36: invalid klee_assume call (provably false)
KLEE: NOTE: now ignoring this error at this location

KLEE: done: total instructions = 137
KLEE: done: completed paths = 5
KLEE: done: generated tests = 5

% ls klee-last
assembly.ll   run.stats            test000002.ktest     test000004.ktest
info          test000001.ktest     test000002.pc        test000005.ktest
messages.txt  test000001.pc        test000002.user.err  warnings.txt
run.istats    test000001.user.err  test000003.ktest
\end{lstlisting}

Первые две ошибки это о \TT{klee\_assume()}.
Это входные значения, на которых застряли вызовы \TT{klee\_assume()}.
Мы можем игнорировать их, либо же, из любопытства, посмотреть, что там:

\begin{lstlisting}
% ktest-tool --write-ints klee-last/test000001.ktest
ktest file : 'klee-last/test000001.ktest'
args       : ['strcmp.bc']
num objects: 2
object    0: name: b'input1'
object    0: size: 2
object    0: data: b'\x00\x00'
object    1: name: b'input2'
object    1: size: 2
object    1: data: b'\x00\x00'

% ktest-tool --write-ints klee-last/test000002.ktest
ktest file : 'klee-last/test000002.ktest'
args       : ['strcmp.bc']
num objects: 2
object    0: name: b'input1'
object    0: size: 2
object    0: data: b'a\xff'
object    1: name: b'input2'
object    1: size: 2
object    1: data: b'\x00\x00'
\end{lstlisting}

Остальные файлы это входные значения для каждого пути в моей реализации \TT{strcmp()}:

\begin{lstlisting}
% ktest-tool --write-ints klee-last/test000003.ktest
ktest file : 'klee-last/test000003.ktest'
args       : ['strcmp.bc']
num objects: 2
object    0: name: b'input1'
object    0: size: 2
object    0: data: b'b\x00'
object    1: name: b'input2'
object    1: size: 2
object    1: data: b'c\x00'

% ktest-tool --write-ints klee-last/test000004.ktest
ktest file : 'klee-last/test000004.ktest'
args       : ['strcmp.bc']
num objects: 2
object    0: name: b'input1'
object    0: size: 2
object    0: data: b'c\x00'
object    1: name: b'input2'
object    1: size: 2
object    1: data: b'a\x00'

% ktest-tool --write-ints klee-last/test000005.ktest
ktest file : 'klee-last/test000005.ktest'
args       : ['strcmp.bc']
num objects: 2
object    0: name: b'input1'
object    0: size: 2
object    0: data: b'a\x00'
object    1: name: b'input2'
object    1: size: 2
object    1: data: b'a\x00'
\end{lstlisting}

3-й это когда первый аргумент (``b'') меньше второго (``c'').
4-й это наоборот (``c'' и ``a'').
5-й это когда они равны (``a'' и ``a'').

Используя эти 3 теста, мы получаем полное покрытие (coverage) для нашей реализации \TT{strcmp()}.

