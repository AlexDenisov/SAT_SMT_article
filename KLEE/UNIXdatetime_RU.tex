\subsection{Дата и время в UNIX}

Дата и время в UNIX\footnote{\url{https://en.wikipedia.org/wiki/Unix_time}} это число секунд прошедших с
1-Jan-1970 00:00 UTC.
Ф-ция gmtime() в Си/Си++ используется для декодирования этого значения в строку, понятную человеку.

Вот фрагмент кода, который я скопипастил из древней версии ОС Minix:
(\url{http://www.cise.ufl.edu/~cop4600/cgi-bin/lxr/http/source.cgi/lib/ansi/gmtime.c}) и немного переработал:

\lstinputlisting[numbers=left]{KLEE/klee_time1.c}

Попробуем:

\begin{lstlisting}
% clang -emit-llvm -c -g klee_time1.c
...

% klee klee_time1.bc
KLEE: output directory is "/home/klee/klee-out-107"
KLEE: WARNING: undefined reference to function: printf
KLEE: ERROR: /home/klee/klee_time1.c:86: external call with symbolic argument: printf
KLEE: NOTE: now ignoring this error at this location
KLEE: ERROR: /home/klee/klee_time1.c:83: ASSERTION FAIL: 0
KLEE: NOTE: now ignoring this error at this location

KLEE: done: total instructions = 101579
KLEE: done: completed paths = 1635
KLEE: done: generated tests = 2
\end{lstlisting}

Ух ты, на строке 83 сработал assert(), почему?
Посмотрим, какое значение UNIX-времени привело к этому:

\begin{lstlisting}
% ls klee-last | grep err
test000001.exec.err
test000002.assert.err

% ktest-tool --write-ints klee-last/test000002.ktest
ktest file : 'klee-last/test000002.ktest'
args       : ['klee_time1.bc']
num objects: 1
object    0: name: b'time'
object    0: size: 4
object    0: data: 978278400
\end{lstlisting}

Попробуем декодировать это значение используя утилиту date в UNIX:

\begin{lstlisting}
% date -u --date='@978278400'
Sun Dec 31 16:00:00 UTC 2000
\end{lstlisting}

После изучения, я нашел что переменная \TT{month} может содержать неверное значение 12 (для которого максимальное это 11,
для декабря), 
потому что макрос LEAPYEAR() должен принимать на вход год как 2000, а не как 100.
Так что пока я переписывал ф-цию, я сделал ошибку, и KLEE нашла её!

Просто интересно, что будет если я заменю switch() на массив строк, как это обычно пишется в кратком коде на Си/Си++?

\begin{lstlisting}
	...

const char *_months[] =
{
	"January", "February", "March",
	"April", "May", "June",
	"July", "August", "September",
	"October", "November", "December"
};

	...

	while (dayno >= _ytab[LEAPYEAR(year)][month])
	{
		dayno -= _ytab[LEAPYEAR(year)][month];
		month++;
	}
	
	char *s=_months[month];

	printf ("%04d-%s-%02d %02d:%02d:%02d\n", YEAR0+year, s, dayno+1, hour, minutes, seconds);
	printf ("week day: %s\n", _days[wday]);	
	
	...

\end{lstlisting}

KLEE обнаруживает попытку прочитать за границами массива:

\begin{lstlisting}
% klee klee_time2.bc
KLEE: output directory is "/home/klee/klee-out-108"
KLEE: WARNING: undefined reference to function: printf
KLEE: ERROR: /home/klee/klee_time2.c:69: external call with symbolic argument: printf
KLEE: NOTE: now ignoring this error at this location
KLEE: ERROR: /home/klee/klee_time2.c:67: memory error: out of bound pointer
KLEE: NOTE: now ignoring this error at this location

KLEE: done: total instructions = 101716
KLEE: done: completed paths = 1635
KLEE: done: generated tests = 2
\end{lstlisting}

Это то же самое UNIX-время, которое мы уже видели:

\begin{lstlisting}
% ls klee-last | grep err
test000001.exec.err
test000002.ptr.err

% ktest-tool --write-ints klee-last/test000002.ktest
ktest file : 'klee-last/test000002.ktest'
args       : ['klee_time2.bc']
num objects: 1
object    0: name: b'time'
object    0: size: 4
object    0: data: 978278400
\end{lstlisting}

Так что, если этот фрагмент кода может быть выполнен на удаленном компьютере, с этим входным значением
(\textit{input of death}),
Так можно свалить процесс (хотя и с какой-то удачей).\\
\\
ОК, теперь я исправляю ошибку, перемещая выражение где отнимается год на строку 43, и теперь посмотрим,
какое UNIX-время соответствует некоторой красивой дате вроде 2022-February-2?

\lstinputlisting[numbers=left]{KLEE/klee_time3.c}

\begin{lstlisting}
% clang -emit-llvm -c -g klee_time3.c
...

% klee klee_time3.bc
KLEE: output directory is "/home/klee/klee-out-109"
KLEE: WARNING: undefined reference to function: klee_assert
KLEE: WARNING ONCE: calling external: klee_assert(0)
KLEE: ERROR: /home/klee/klee_time3.c:47: failed external call: klee_assert
KLEE: NOTE: now ignoring this error at this location

KLEE: done: total instructions = 101087
KLEE: done: completed paths = 1635
KLEE: done: generated tests = 1635

% ls klee-last | grep err
test000587.external.err

% ktest-tool --write-ints klee-last/test000587.ktest
ktest file : 'klee-last/test000587.ktest'
args       : ['klee_time3.bc']
num objects: 1
object    0: name: b'time'
object    0: size: 4
object    0: data: 1645488640

% date -u --date='@1645488640'
Tue Feb 22 00:10:40 UTC 2022
\end{lstlisting}

Успешно нашли, но часы/минуты/секунды выглядят как случайные --- они и правда случайные, потому что KLEE удовлетворило
все констрайнты нами добавленные, ничего более.
Мы ведь не просили выставить часы/минуты/секунды в нули.

Добавим также констрайнты для часом/минут/секунд:

\begin{lstlisting}
	...

	if (YEAR0+year==2022 && month==1 && dayno+1==22 && hour==22 && minutes==22 && seconds==22)
		klee_assert(0);
	
	...
\end{lstlisting}

Запустим и проверим \dots

% FIXME:
\begin{lstlisting}
% ktest-tool --write-ints klee-last/test000597.ktest
ktest file : 'klee-last/test000597.ktest'
args       : ['klee_time3.bc']
num objects: 1
object    0: name: b'time'
object    0: size: 4
object    0: data: 1645568542

% date -u --date='@1645568542'
Tue Feb 22 22:22:22 UTC 2022
\end{lstlisting}

Теперь всё точно.

Да, конечно, в библиотеках Си/Си++ есть ф-ции для кодирования строки с датой в UNIX-время, но то что мы тут получили,
это KLEE работающий как \textit{антипод} декодирующей ф-ции, \textit{инверсная ф-ция} в каком-то смысле.
