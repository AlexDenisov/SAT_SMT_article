\subsection{strtodx() из RetroBSD}

Нашел эту ф-цию в RetroBSD:
\url{https://github.com/RetroBSD/retrobsd/blob/master/src/libc/stdlib/strtod.c}.
Она конвертирует строку в число с плавающей точкой для заданной системы исчисления.

\lstinputlisting[numbers=left]{KLEE/strtodx.c}
( \url{https://github.com/dennis714/SAT_SMT_article/blob/master/KLEE/strtodx.c} )

Интересно, KLEE не поддерживает арифметику с плавающей точкой, но тем не менее, что-то нашел:

\begin{lstlisting}
...

KLEE: ERROR: /home/klee/klee_test.c:202: memory error: out of bound pointer

...

% ktest-tool klee-last/test003483.ktest
ktest file : 'klee-last/test003483.ktest'
args       : ['klee_test.bc']
num objects: 1
object    0: name: b'buf'
object    0: size: 10
object    0: data: b'-.0E-66\x00\x00\x00'
\end{lstlisting}

Как видно, строка ``-.0E-66'' приводит к чтению из массива за его пределами, в строке 202.
Во время дальнейшего изучения, я обнаружил что массив \TT{powersOf10[]} слишком короткий:
был прочитан 6-й элемент (считая с нулевого).
И мы видим что часть массива закомментирована (строка 79)!
Вероятно, чья-то ошибка?

