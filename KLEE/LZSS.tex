\subsection{LZSS decompressor}

I've googled for a very simple LZSS decompressor and landed at this page:
\url{http://www.opensource.apple.com/source/boot/boot-132/i386/boot2/lzss.c}.

Let's pretend, we're looking at unknown compressing algorithm with no compressor available.
Will it be possible to construct a compressed piece of data so that decompressor would generate data we need?

Here is my first experiment:

\lstinputlisting{KLEE/klee_lzss1.c}

What I did is changing size of ring buffer from 4096 to 32, because if bigger, KLEE consumes all RAM it can.
I found that KLEE can live with that small buffer.
I also decreased \TT{COMPRESSED\_LEN} gradually to check, whether KLEE would find compressed piece of data, and it did:

% FIXME:
\begin{lstlisting}
\$ clang -emit-llvm -c -g klee_lzss.c
...

\$ time klee klee_lzss.bc
KLEE: output directory is "/home/klee/klee-out-7"
KLEE: WARNING: undefined reference to function: klee_assert
KLEE: ERROR: /home/klee/klee_lzss.c:122: invalid klee_assume call (provably false)
KLEE: NOTE: now ignoring this error at this location
KLEE: ERROR: /home/klee/klee_lzss.c:47: memory error: out of bound pointer
KLEE: NOTE: now ignoring this error at this location
KLEE: ERROR: /home/klee/klee_lzss.c:37: memory error: out of bound pointer
KLEE: NOTE: now ignoring this error at this location
KLEE: WARNING ONCE: calling external: klee_assert(0)
KLEE: ERROR: /home/klee/klee_lzss.c:124: failed external call: klee_assert
KLEE: NOTE: now ignoring this error at this location

KLEE: done: total instructions = 41417919
KLEE: done: completed paths = 437820
KLEE: done: generated tests = 4

real    13m0.215s
user    11m57.517s
sys     1m2.187s

\$ ls klee-last | grep err
test000001.user.err
test000002.ptr.err
test000003.ptr.err
test000004.external.err

\$ ktest-tool --write-ints klee-last/test000004.ktest
ktest file : 'klee-last/test000004.ktest'
args       : ['klee_lzss.bc']
num objects: 1
object    0: name: b'input'
object    0: size: 15
object    0: data: b'\xffBuffalo \x01b\x0f\x03\r\x05'
\end{lstlisting}

% FIXME tilde
KLEE consumed ~1GB of RAM and worked for ~15 minutes (on my Intel Core i3-3110M 2.4GHz notebook), 
but here it is, a 15 bytes which, if decompressed by our copypasted algorithm, will result in desired text!

During my experimentation, I've found that KLEE can do even more cooler thing, to find out size of compressed piece of data:

% FIXME:
\begin{lstlisting}
int main()
{
	uint8_t input[24];
	uint8_t plain[24];
	uint32_t size;
  
	klee_make_symbolic(input, sizeof input, "input");
	klee_make_symbolic(&size, sizeof size, "size");
	
	decompress_lzss(plain, input, size);

	for (int i=0; i<23; i++)
		klee_assume (plain[i]=="Buffalo buffalo Buffalo"[i]);

	klee_assert(0);
	
	return 0;
}
\end{lstlisting}

... but then KLEE works much slower, consumes much more RAM and I had success only with even smaller pieces of desired text.

So how LZSS works? Without peeking into Wikipedia, we can say that: 
if LZSS compressor observes some data it already had, it replaces the data with a link to some place in past with size. 
If it observes something yet unseen, it puts data as is.
This is theory.
This is indeed what we've got. Desired text is three ``Buffalo'' words, the first and the last are equivalent, but the second is \textit{almost} equivalent, 
differing with first by one character.

That's what we see:

% FIXME: colored Buffalo, ``b'', slashes
\begin{lstlisting}
'\xffBuffalo \x01b\x0f\x03\r\x05'
\end{lstlisting}

Here is some control byte (0xff), ``Buffalo'' word is placed \textit{as is}, then another control byte (0x01), 
then we see beginning of the second word (``b'') and more
control bytes, perhaps, links to the beginning of the buffer.
These are command to decompressor, like, in plain English, ``copy data from the buffer we've already done, from that place to that place'', etc.

Interesting, is it possible to meddle into this piece of compressed data?
Out of whim, can we force KLEE to find a compressed data, where not just ``b'' character has been placed \textit{as is},
but also the second character of the word, i.e., ``bu''?

I've modified main() function by adding \TT{klee\_assume()}: now the 11th byte of input (compressed) data (right after ``b'' byte) should have ``u''.
I has no luck with 15 byte of compressed data, so I increased it to 16 bytes:

% FIXME:
\begin{lstlisting}
int main()
{
#define COMPRESSED_LEN 16
	uint8_t input[COMPRESSED_LEN];
	uint8_t plain[24];
	uint32_t size=COMPRESSED_LEN;
  
	klee_make_symbolic(input, sizeof input, "input");
	
	klee_assume(input[11]=='u');
	
	decompress_lzss(plain, input, size);

	for (int i=0; i<23; i++)
		klee_assume (plain[i]=="Buffalo buffalo Buffalo"[i]);

	klee_assert(0);
	
	return 0;
}
\end{lstlisting}

% FIXME: umlaut
... and voila: KLEE found a compressed piece of data which satisfied our whimsical constraint:

% FIXME:
\begin{lstlisting}
\$ time klee klee_lzss.bc
KLEE: output directory is "/home/klee/klee-out-9"
KLEE: WARNING: undefined reference to function: klee_assert
KLEE: ERROR: /home/klee/klee_lzss.c:97: invalid klee_assume call (provably false)
KLEE: NOTE: now ignoring this error at this location
KLEE: ERROR: /home/klee/klee_lzss.c:47: memory error: out of bound pointer
KLEE: NOTE: now ignoring this error at this location
KLEE: ERROR: /home/klee/klee_lzss.c:37: memory error: out of bound pointer
KLEE: NOTE: now ignoring this error at this location
KLEE: WARNING ONCE: calling external: klee_assert(0)
KLEE: ERROR: /home/klee/klee_lzss.c:99: failed external call: klee_assert
KLEE: NOTE: now ignoring this error at this location

KLEE: done: total instructions = 36700587
KLEE: done: completed paths = 369756
KLEE: done: generated tests = 4

real    12m16.983s
user    11m17.492s
sys     0m58.358s

\$ ktest-tool --write-ints klee-last/test000004.ktest
ktest file : 'klee-last/test000004.ktest'
args       : ['klee_lzss.bc']
num objects: 1
object    0: name: b'input'
object    0: size: 16
object    0: data: b'\xffBuffalo \x13bu\x10\x02\r\x05'
\end{lstlisting}

So now we find a piece of data where two strings are placed \textit{as is}: ``Buffalo'' and ``bu''.

% FIXME: colored Buffalo and bu
\begin{lstlisting}
'\xffBuffalo \x13bu\x10\x02\r\x05'
\end{lstlisting}

Both pieces of compressed data, if feeded into our copypasted function, produce ``Buffalo buffalo Buffalo'' text string.

Please note, I still have no access to LZSS compressor code, and I didn't get into LZSS decompressor details yet.

Unfortunately, things are not that cool: 
KLEE is very slow and I had success only with small pieces of text, and also ring buffer size had to be decreased significantly
(original LZSS decompressor with ring buffer of 4096 bytes cannot decompress correctly what we found).

Nevertheless, it's very impressive, taking into account the fact that we're not getting into internals of this specific LZSS decompressor.
Once more time, we've created \textit{antipode} of decompressor, or \textit{inverse function}.

Also, as it seems, KLEE isn't very good so far with decompression algorithms (but who's good then?).
I've also tried various JPEG/PNG/GIF decoders (which, of course, has decompressors), starting with simplest possible, and KLEE had stuck.

