\subsection{Обратная ф-ция base64-декодера}

Для KLEE нет никаких проблем реконструировать входную base64-строку, имея только код декодировщик base64, без соответствующего
кодировщика.
Я скопипастил код из
\url{http://www.opensource.apple.com/source/QuickTimeStreamingServer/QuickTimeStreamingServer-452/CommonUtilitiesLib/base64.c}.

Мы добавляем констрайнты (строки 84, 85) так что выходной буфер будет иметь байты от 0 до 15.
Мы также говорим KLEE что ф-ция Base64decode() должна вернуть 16 (т.е., размер выходного буфера в байтах, строка 82).

\lstinputlisting[numbers=left]{KLEE/klee_base64.c}

\begin{lstlisting}
% clang -emit-llvm -c -g klee_base64.c
...

% klee klee_base64.bc
KLEE: output directory is "/home/klee/klee-out-99"
KLEE: WARNING: undefined reference to function: klee_assert
KLEE: ERROR: /home/klee/klee_base64.c:99: invalid klee_assume call (provably false)
KLEE: NOTE: now ignoring this error at this location
KLEE: WARNING ONCE: calling external: klee_assert(0)
KLEE: ERROR: /home/klee/klee_base64.c:104: failed external call: klee_assert
KLEE: NOTE: now ignoring this error at this location
KLEE: ERROR: /home/klee/klee_base64.c:85: memory error: out of bound pointer
KLEE: NOTE: now ignoring this error at this location
KLEE: ERROR: /home/klee/klee_base64.c:81: memory error: out of bound pointer
KLEE: NOTE: now ignoring this error at this location
KLEE: ERROR: /home/klee/klee_base64.c:65: memory error: out of bound pointer
KLEE: NOTE: now ignoring this error at this location

...
\end{lstlisting}

Интересно посмотреть на вторую ошиюку, где сработала \TT{klee\_assert()}:

\begin{lstlisting}
% ls klee-last | grep err
test000001.user.err
test000002.external.err
test000003.ptr.err
test000004.ptr.err
test000005.ptr.err

% ktest-tool --write-ints klee-last/test000002.ktest
ktest file : 'klee-last/test000002.ktest'
args       : ['klee_base64.bc']
num objects: 1
object    0: name: b'input'
object    0: size: 32
object    0: data: b'AAECAwQFBgcICQoLDA0OD4\x00\xff\xff\xff\xff\xff\xff\xff\xff\x00'
\end{lstlisting}

Это действительно настоящая base64-строка, оканчивающаяся нулевым байтом, как и требуется по стандартам Си/Си++.
Последний нулевой байт на месте 31-го (считая с нулевого) это дело наших рук: так KLEE будет выдавать меньшее количество
ошибок.

Строка в base64 действительно корректна:

\begin{lstlisting}
% echo AAECAwQFBgcICQoLDA0OD4 | base64 -d | hexdump -C
base64: invalid input
00000000  00 01 02 03 04 05 06 07  08 09 0a 0b 0c 0d 0e 0f  |................|
00000010
\end{lstlisting}

Утилита для декодирования base64 из Linux которую я только что запустил, ругается: ``invalid input'' --- 
это означает что строка неверно дополнена выровнивающими символами (``='').
Дополним вручную, и декодер больше не будет ругаться:

\begin{lstlisting}
% echo AAECAwQFBgcICQoLDA0OD4== | base64 -d | hexdump -C
00000000  00 01 02 03 04 05 06 07  08 09 0a 0b 0c 0d 0e 0f  |................|
00000010
\end{lstlisting}

Причина, по которой наша base64-строка не была дополнена, в том, что декодеры base64 обычно игнорируют эти символы (``='')
в конце.
Другими словами, они не тербуют их, так же, как и наш декодер.
Так что, символы для выравнивания остались незамеченными для KLEE.

И снова мы сделали \textit{антипод} или \textit{инверсную функцию} декодера base64.

