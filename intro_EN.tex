\section{This is a draft!}

This is very early draft, but still can be interesting for someone.

Latest version is always available at \url{http://yurichev.com/writings/SAT_SMT_draft-EN.pdf}.
Russin version is at \url{http://yurichev.com/writings/SAT_SMT_draft-RU.pdf}.

Current text version: \today{}.

For news about updates, you may subscribe my 
twitter\footnote{\url{https://twitter.com/yurichev}}, 
facebook\footnote{\url{https://www.facebook.com/dennis.yurichev.5}}, 
or github repo\footnote{\url{https://github.com/dennis714/SAT_SMT_article}}.

\section{Thanks}

Leonardo de Moura and Nikolaj Bjorner, for help.

\section{Introduction}

\ac{SAT}/\ac{SMT} solvers can be viewed as solvers of huge systems of equations.
The difference is that \ac{SMT} solvers takes systems in arbitrary format,
while \ac{SAT} solvers are limited to boolean equations in \ac{CNF} form.

A lot of real world problems can be represented as problems of solving system of equations.

\section{Is it a hype? Yet another fad?}

Some people say, this is just another hype.
No, \ac{SAT} is old enough and fundamental to \ac{CS}.
The reason of increased interest to it is that computers gets faster over the last couple decades,
so there are attempts to solve old problems using \ac{SAT}/\ac{SMT}, which were inaccessible in past.

